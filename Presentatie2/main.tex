%voeg 'backroundtitleslede' voor titleslidebackround
%voeg 'backround' toe voor al de rest



\documentclass[mathserif,11pt]{beamer}

\mode<presentation> {
\usetheme{Singapore}
\usecolortheme{seagull}
%\setbeamertemplate{headline}{}
\setbeamertemplate{footline}[page number] % To replace the footer line in all slides with a simple slide count uncomment this line
\setbeamerfont{frametitle}{size=\Large}
\setbeamerfont{title}{size=\Large}
\setbeamertemplate{navigation symbols}{} % To remove the navigation symbols from the bottom of all slides uncomment this line
\useoutertheme{infolines}
%\usefonttheme[⟨onlylarge⟩]{structuresmallcapsserif}

}
%\usepackage{animate}
\usepackage{graphicx}
%\usepackage{movie15}
\usepackage{lmodern} % Allows including images
\usepackage{booktabs}
\usepackage{textpos}
%\usepackage{media9}
\usepackage{url}
%\usepackage{multimedia}
\usepackage[font={scriptsize,it}]{caption}
\usepackage{subfigure}
\usepackage[export]{adjustbox}
\usepackage{color,xcolor,ucs}
\usepackage{geometry}
\usepackage{enumerate}
%\usepackage{ocgx2}
\usefonttheme{serif}
\newcommand{\tabitem}{%
  \usebeamertemplate{itemize item}\hspace*{\labelsep}}
%----------------------------------------------------------------------------------------
%	TITLE PAGE
%----------------------------------------------------------------------------------------

\makeatletter
\newcommand{\footlineextra}[1]{\gdef\insertfootlineextra{#1}}
\newbox\footlineextrabox


\defbeamertemplate*{footline extra}{default}{
    \begin{beamercolorbox}[ht=2.25ex,dp=1ex,leftskip=\Gm@lmargin]{footline extra}
    \insertfootlineextra
    %\par\vspace{2.5pt}
    \end{beamercolorbox}
}

\addtobeamertemplate{footline}{%
    \setbox\footlineextrabox=\vbox{\usebeamertemplate*{footline extra}}
    \vskip -\ht\footlineextrabox
    \vskip -\dp\footlineextrabox
    \box\footlineextrabox%
}
%{}


\let\beamer@original@frame=\frame
\def\frame{\gdef\insertfootlineextra{}\beamer@original@frame}
\footlineextra{}
\makeatother

\title[Katholieke Universiteit Leuven]{\small Bachelor Thesis 2016 \\ \vspace{0.2cm} \Huge Radiosensitization \\ using gold nanoparticles}
\author[L. Deceuninck en H. Verhoeven]{Lies Deceuninck en Hannelore Verhoeven} % Your name
\date


\begin{document}

%\usebackgroundtemplate{%
 %\includegraphics[width=\paperwidth
 %]{gfx/Back1.png}}
\begin{frame}[plain]
\maketitle
\vspace{-1cm}
\begin{columns}[c]
\column{.5\textwidth}
\begin{center}
\begin{tabular}{l l}
Supervisors: & Bert De Roo \\ % Partner names
& Mattias Vervaele \\
Professor: & Jean-Pierre Locquet % Instructor/supervisor
\end{tabular}
\end{center}
\column{.4\textwidth}
\end{columns}
\end{frame}
\usebackgroundtemplate{%
 \includegraphics[width=\paperwidth]
 {gfx/first}}
\begin{frame}[plain]

\end{frame}
\setcounter{framenumber}{0}
\usebackgroundtemplate{}

%----------------------------------------------------------------------------------------
%	PRESENTATION SLIDES
%----------------------------------------------------------------------------------------
\section{Introduction}
\subsection{Cancer treatment}
%------------------------------------------------
	\begin{frame}{DNA damage using ionizing radiation}
	 \footlineextra{Source: https://chem.kuleuven.be/veiligheid/info/ioniserende-st.htm}

		\begin{columns}[c]

		\column{.4\textwidth} % Left column and width

   			\begin{itemize}
   				\item Chemotherapy
    			\item Surgery
					\item \textbf{Radiation therapy}

   			\end{itemize}
				\qquad Energy $\sim$ MeV

   		\column{.5\textwidth}

			\begin{figure}[tb]
    			\centering
    			\includegraphics[width=5cm]{gfx/direct2.png}
    			\label{fig:direct}
			\end{figure}

    \end{columns}

	\end{frame}




	\begin{frame}{Radiosensitization of cancer cells with \\ gold nanoparticles (GNP) \footnotesize E $\sim $ keV}
	\footlineextra{Source: https://en.wikibooks.org/wiki/Basic_Physics_of_Digital_Radiography/The_Patient}
	%\centering
	\vspace{.3cm}
	\begin{columns}[c]
    \centering
		\column{.35\textwidth}
			Photoelectric absorption
			%\vspace{-.8cm}
				\begin{figure}[tb]
						\centering
    				\includegraphics[width=3.5cm]{gfx/photoel2.png}
    				\label{fig:photoel}
				\end{figure}
				%\vspace{-.4cm}
		\column{.35\textwidth}
	\centering
			Compton effect
			%\vspace{-.8cm}
				\begin{figure}[tb]
    				\centering
    				\includegraphics[width=3.5cm]{gfx/compton2.png}
    				\label{fig:compton}
				\end{figure}
		\end{columns}
\end{frame}

	\begin{frame}{Targeting of the GNP to the tumor}
	\begin{columns}[c]
	\column{0.5\textwidth}
	\centering
	\textbf{Passive targeting} \\
	 PEG coating
		\begin{figure}[tb]
    		\centering
    			\includegraphics[width=0.8\textwidth]{gfx/ptargeting.png}
    			\label{fig:ptargeting}
		\end{figure}
	\column{0.5\textwidth}
	\centering
	 \textbf{Active targeting} \\
	 Antibodies
	\begin{figure}[tb]
    			\centering
    			\includegraphics[width=0.8\textwidth]{gfx/atargeting.png}
    			\label{fig:atargeting}
			\end{figure}
	\end{columns}

	\end{frame}

  \begin{frame}[plain]{}
  %\frametitle{Overview project}
\footlineextra{Source: A. Chaudhary , A. Gupta , S. Khan and C. Kanti Nandi \emph{Phys. Chem. Chem. Phys.}, 2014 \textbf{16}}
%\vspace{0.5cm}
\centering
\Large Overview Project\\
\vspace{0.3cm}
\large Radiosensitization of cancer cells \\using gold nanoparticles
\begin{columns}[c]
\normalsize

	\column{.33\textwidth}
	%\tableofcontents
  \begin{enumerate}
    \item Synthesis
    \item Characterization
		\begin{enumerate}[a.]
			\item TEM
			\item UV-Vis
			\item DLS
\end{enumerate}
    \item Radiosensitization
  \end{enumerate}
	\column{.6\textwidth}

\begin{figure}[tb]
	\centering
	\includegraphics[width=0.9\textwidth]{gfx/Inhoud}
\end{figure}
	\end{columns}

\end{frame}
\setcounter{framenumber}{3}

%------------------------------------------------
%------------------------------------------------
\section{Synthesis}
%------------------------------------------------
%------------------------------------------------

%------------------------------------------------
\subsection{Chemical protocol}
%------------------------------------------------

	\begin{frame}{Reduction of gold ions to form GNP}

		\begin{figure}[!h]
			\centering
			\includegraphics[width=0.8\textwidth]{gfx/GNPReduction}
		\end{figure}
    \vspace{-2.2cm}
    \Large
      \begin{center}
          Reducing \\agent
      \end{center}

		\vspace{1cm}
\normalsize
		\begin{columns}[c]
		\column{.5\textwidth}

		Gold ions: HAuCl$_4$ solution\\
		Reducing agent: Na$_3$C$_6$H$_5$O$_7$
		\column{.15\textwidth}
		\end{columns}

    \end{frame}
    \setcounter{framenumber}{3}

	\begin{frame}[t]
	\frametitle{Reduction of gold ions to form GNP}
  \vspace{1cm}
  \begin{columns}[c r]
    	\column{.35\textwidth}
      \vspace{0.5cm}
      \Large
      OH$^-\qquad $ C$_6$H$_5$O$_7^-$
      	\column{.4\textwidth}
        \vspace{1cm}
        \Large
        $\qquad \qquad $ H$_2$O + CO$_2$

  \end{columns}

		\vspace{-1.7cm}
		\begin{figure}[h!]
			\centering
			\includegraphics[width=0.9\textwidth]{gfx/Reduction}
		\end{figure}
      \vspace{-2.5cm}
      \begin{columns}
        \column{.15\textwidth}
        \column{.54\textwidth}
          \Large
          Reduction
      \end{columns}

		\centering
    \vspace{2cm}
  %  \vspace{-0.5cm}
		$6$Au$^3^+$ $+$ C$_6$H$_5$O$_7^{3^-}$ $+$ $15$OH$^-\rightarrow$ Au $+$ $6$CO$_2$ $+$ $10$H$_2$O

	\end{frame}
%------------------------------------------------
\subsection{Size GNP}
%------------------------------------------------
	\begin{frame}[t]
	\frametitle{The amount of citrate controls the size}
		\begin{figure}[tb]
    		\centering
    		\includegraphics[width=0.8\textwidth]{gfx/SizeGNP}
    	\end{figure}
      \vspace{-2cm}
      %\centering
$\qquad \qquad$ $15$ nm $\qquad \qquad \quad$ $30$ nm $\qquad \qquad \qquad \quad$   $45$ nm\\
$\qquad \qquad$ $2.5 $ml $\qquad \qquad \quad$ $1.24$ ml $\qquad \qquad \qquad \quad$ $0.8$ ml
\vspace{1cm}
      \begin{columns}[c]
		\column{.5\textwidth}
			Citrate $1$\%\\
    		\small $100$ml HAuCl$_4$ $0.01$\%
		\column{.15\textwidth}
		\end{columns}
\end{frame}


%------------------------------------------------
\subsection{Functionalization}
%------------------------------------------------


\begin{frame}[t]
\frametitle{PEG for targeting and stabilization}

    \begin{figure}[tb]
    	\centering
    	\includegraphics[width=0.7\textwidth]{gfx/PEG}
    \end{figure}
    \vspace{-1.4cm}
   $\quad\qquad \qquad$  Ethylene glycol $\qquad \quad$ Polyethylene glycol
    %\vspace{1cm}
    \begin{columns}
    	\column{.5\textwidth}
    	\begin{figure}[tb]
    		\centering
    		\includegraphics[width=0.7\textwidth]{gfx/PEG2}
    	\end{figure}
    	\centering
      $1$k, $5$k, $10$k, $20$k
    	\column{.5\textwidth}
    	\begin{figure}[tb]
    		\centering
    		\includegraphics[width=0.8\textwidth]{gfx/ptargeting.png}
    	\end{figure}

    \end{columns}
\end{frame}

%------------------------------------------------
\section{Characterization}
%------------------------------------------------

%------------------------------------------------
\subsection{Transmission electron microscope (TEM)}
%------------------------------------------------
\begin{frame}{TEM image analysis to determine \\the core diameter}

				\begin{overprint}
    		\only<1>{
					\begin{columns}
					\column{.5\textwidth}
        		\begin{figure}[tb]
            		\centering
            		\includegraphics[width=4.5cm]{gfx/GNP_8}
        		\end{figure}
					\column{.5\textwidth}
        		\begin{figure}[tb]
            		\centering
            		\includegraphics[width=6cm]{gfx/TEM_8}
        		\end{figure}
					\end{columns}
       	 	}
        	\pause
    		\only<2>{
				\vspace{-.45cm}
					\begin{columns}
					\column{.5\textwidth}
        		\begin{figure}[tb]
            		\centering
            		\includegraphics[width=4.5cm]{gfx/GNP_8}
        		\end{figure}
					\column{.5\textwidth}
\begin{table}
  \begin{center}
  \begin{tabular}{ c | c }
    Exp. Size (nm) & Size (nm)\\
    \toprule
    $15$ & $12.98 \pm 0.23$\\
     & $2.99 \pm 0.16$\\
    $30$ & $18.29 \pm 0.23$\\
    $45$ & $46.75 \pm 0.47$
  \end{tabular}
  \end{center}
\end{table}
				\end{columns}
					}

			\end{overprint}
\end{frame}

%------------------------------------------------
\subsection{Ultraviolet and visible spectroscopy (UV-Vis)}
%------------------------------------------------

\begin{frame}{Absorption measurements (UV-Vis) \\to determine the relative size}
				\begin{overprint}
    		\only<1>{
					\begin{columns}
					\column{.55\textwidth}
        		\begin{figure}[tb]
            		\centering
            		\includegraphics[width=6cm]{gfx/plasmon}
        		\end{figure}
					\column{.45\textwidth}
        		\begin{figure}[tb]
            		\centering
            		\includegraphics[width=5.5cm]{gfx/absorption}
        		\end{figure}
					\end{columns}
       	 	}
        	\pause
    		\only<2>{
        		\begin{figure}[tb]
            		\centering
            		\includegraphics[width=12cm]{gfx/vis1}
        		\end{figure}
					}

			\end{overprint}
\end{frame}

\begin{frame}{Absorption measurements (UV-Vis) \\to determine the optimal PEG proportion}
      	\begin{overprint}
    		\only<1>{
				\vspace{0.2cm}
						\centering
						$1$. Add PEG and preform UV-Vis measurement
				\vspace{.2cm}
        		\begin{figure}[tb]
            		\centering
            		\includegraphics[width=12cm]{gfx/vis2.png}
        		\end{figure}

       	 	}
        	\pause
    		\only<2>{
						\centering
						$2$. Add NaCl and preform UV-Vis measurement
        		\begin{figure}[tb]
            		\centering
            		\includegraphics[width=12cm]{gfx/vis3.png}
        		\end{figure}
        	}
        	\pause
    		\only<3>{
						\centering
						bigger size $\rightarrow$ too little PEG\\
						same size $\rightarrow$ enough PEG
				\vspace{-.45cm}
        		\begin{figure}[tb]
            		\centering
            		\includegraphics[width=12cm]{gfx/vis4.png}
        		\end{figure}
						\centering
        	}
			\end{overprint}

\end{frame}

\subsection{Dynamic light scattering (DLS)}
\begin{frame}[t]
\frametitle{Light scattering experiments to determine \\ the hydrodynamic radius (R$_h$)}
\framesubtitle{Important for diffusive properties}
\vspace{-0.7cm}
\begin{columns}
	\column{.3\textwidth}
	\begin{figure}[tb]
		\centering
		\includegraphics[width=0.8\textwidth]{gfx/GNPPEG2}
	\end{figure}
	\column{.6\textwidth}
  \vspace{0.7}
  $\qquad \qquad \qquad$ Correlation:\\
  \vspace{0.3cm}
  $\qquad \qquad \qquad$ $g(\tau)=\frac{\langle I(t)\rangle \langle I(t+\tau)\rangle}{\langle I(t)\rangle^2}$
\end{columns}
\vspace{-0.7cm}
%\vspace{-2cm}
\begin{columns}
    \column{.5\textwidth}
    \begin{figure}[tb]
    	\centering
    	\includegraphics[width=0.9\textwidth]{gfx/DLSint.png}
    \end{figure}
    \column{.5\textwidth}
    \begin{figure}[tb]
    	\centering
    	\includegraphics[width=0.9\textwidth]{gfx/DLScor}
    \end{figure}
\end{columns}
\end{frame}
\begin{frame}{Hydrodynamic radii for the three different GNP \\with different functionalizations}
  \begin{figure}[tb]
    \centering
    \includegraphics[width=0.7\textwidth]{gfx/DlsAll.png}
  \end{figure}
\end{frame}
\section{Radiosensitization}
\subsection{DNA damage analysis}
\begin{frame}{Gel electrophoresis to analyse DNA damage}
  \begin{overprint}
  \only<1>{
    \begin{columns}[b]
    \column{.6\textwidth}
    \begin{figure}
      \includegraphics[width=\textwidth]{gfx/schade.png}
    \end{figure}
    \column{.3\textwidth}
    \vspace{1.5cm}
    Single strand break\\
    Double strand break\\
    \end{columns}
    }
    \pause
  \only<2>{
\begin{figure}
  \includegraphics[width=0.7\textwidth]{gfx/gel2}
\end{figure}
    }
\end{overprint}
\end{frame}

\begin{frame}{DNA damage analysis for $15$nm GNP}
  \begin{figure}[tb]
    \centering
    \includegraphics[width=0.8\textwidth]{gfx/15}
  \end{figure}
\end{frame}

\begin{frame}{DNA damage analysis for $30$nm GNP}
  \begin{figure}[tb]
    \centering
    \includegraphics[width=0.8\textwidth]{gfx/30}
  \end{figure}
\end{frame}

\begin{frame}{DNA damage analysis for $45$nm GNP}
  \begin{figure}[tb]
    \centering
    \includegraphics[width=0.8\textwidth]{gfx/45}
  \end{figure}
\end{frame}

\begin{frame}{Conclusion DNA damage analysis}
\begin{columns}
\column{0.7\textwidth}
\begin{itemize}
    \item Strange results for $1$ k PEG
    \item Curves are not rising (except $30$ nm GNP)
    \item Reference sample has strange form
		\item DNA damage at $0$ Gy
		\item Radiosensitization effect
		\item Smaller PEG $\rightarrow$ more DNA damage
		\item Larger GNP $\rightarrow$ more DNA damage
\end{itemize}
\column{0.2\textwidth}
  \begin{figure}[tb]
    \centering
    \includegraphics[width=\textwidth]{gfx/schade2}
  \end{figure}
\end{columns}
\end{frame}

\begin{frame}{Conclusion}
%\vspace{1cm}
%\begin{columns}
%\column{0.6\textwidth}

\begin{center}
    \begin{tabular}{@{}l@{}}
      \tabitem Functionalization $1$k strange behavior \\
      \tabitem Best results with...\\
        $\qquad$...$5$k functionalization\\
        $\qquad$...largest particles ($45$ nm)\\
    \end{tabular}
  \end{center}
  \centering
$\Rightarrow$ Radiosensitization effect observed

%\column{0.001\textwidth}

%\end{columns}
%\vspace{-2cm}
\begin{figure}[tb]
	\centering
	\includegraphics[width=0.6\textwidth]{gfx/overview}

\end{figure}
\end{frame}

% section characterization (end)
% \begin{frame}{Referenties}


% \begin{thebibliography}{99} % Beamer does not support BibTeX so references must be inserted manually as below
% \setbeamertemplate{bibliography item}[article]
% \bibitem[S. R. Makhsin et all.]{R1} (2012) The effects of size and synthesis methods of gold nanoparticle-conjugated MαHIgG4 for use in an immunochromatographic strip test to detect brugian filariasis
% \newblock  \emph{S. R. Makhsin, K. A. Razak, R. Noordin, N. D. Zakaria and T. S. Chun}, 2012 November 19
% \newblock Universiti Sains Malaysia
% \end{thebibliography}

% \end{frame}

\end{document}
