%voeg 'backroundtitleslede' voor titleslidebackround
%voeg 'backround' toe voor al de rest



\documentclass[mathserif,11pt]{beamer}

\mode<presentation> {
\usetheme{Singapore}
\usecolortheme{seagull}
%\setbeamertemplate{headline}{}
\setbeamertemplate{footline}[page number] % To replace the footer line in all slides with a simple slide count uncomment this line
\setbeamerfont{frametitle}{size=\Large}
\setbeamerfont{title}{size=\Large}
\setbeamertemplate{navigation symbols}{} % To remove the navigation symbols from the bottom of all slides uncomment this line
\useoutertheme{infolines}
%\usefonttheme[⟨onlylarge⟩]{structuresmallcapsserif}

}
%\usepackage{animate}
\usepackage{graphicx}
%\usepackage{movie15}
\usepackage{lmodern} % Allows including images
\usepackage{booktabs}
\usepackage{textpos}
%\usepackage{media9}
\usepackage{url}
%\usepackage{multimedia}
\usepackage[font={scriptsize,it}]{caption}
\usepackage{subfigure}
\usepackage[export]{adjustbox}
\usepackage{color,xcolor,ucs}
\usepackage{geometry}
%\usepackage{ocgx2}
\usefonttheme{serif}

%----------------------------------------------------------------------------------------
%	TITLE PAGE
%----------------------------------------------------------------------------------------

\makeatletter
\newcommand{\footlineextra}[1]{\gdef\insertfootlineextra{#1}}
\newbox\footlineextrabox


\defbeamertemplate*{footline extra}{default}{
    \begin{beamercolorbox}[ht=2.25ex,dp=1ex,leftskip=\Gm@lmargin]{footline extra}
    \insertfootlineextra
    %\par\vspace{2.5pt}
    \end{beamercolorbox}
}

\addtobeamertemplate{footline}{%
    \setbox\footlineextrabox=\vbox{\usebeamertemplate*{footline extra}}
    \vskip -\ht\footlineextrabox
    \vskip -\dp\footlineextrabox
    \box\footlineextrabox%
}
{}


\let\beamer@original@frame=\frame
\def\frame{\gdef\insertfootlineextra{}\beamer@original@frame}
\footlineextra{}
\makeatother

\title[Katholieke Universiteit Leuven]{\small Bachelor Thesis 2016\\\vspace{0.2cm} \Huge Radiosensitization
\\ using gold nanoparticles}
\author[L. Deceuninck en H. Verhoeven]{Lies Deceuninck en Hannelore Verhoeven} % Your name
\date


\begin{document}

%\usebackgroundtemplate{%
 %\includegraphics[width=\paperwidth
 %]{gfx/Back1.png}}
\begin{frame}[plain]
\maketitle
\vspace{-1cm}
\begin{columns}[c]
\column{.5\textwidth}
\begin{center}
\begin{tabular}{l l}
Assistents: & Bert De Roo \\ % Partner names
& Mattias Vervaele \\
Professor: & Chris Van Haesendonck % Instructor/supervisor
\end{tabular}
\end{center}
\column{.4\textwidth}
\end{columns}
\end{frame}
\setcounter{framenumber}{0}
%\usebackgroundtemplate{}

%----------------------------------------------------------------------------------------
%	PRESENTATION SLIDES
%----------------------------------------------------------------------------------------
\section{Introduction}
\subsection{Cancer treatment}
%------------------------------------------------
	\begin{frame}{DNA damage using ionizing radiation}
	 \footlineextra{Source: https://chem.kuleuven.be/veiligheid/info/ioniserende-st.htm}

		\begin{columns}[c]

		\column{.4\textwidth} % Left column and width

   			\begin{itemize}
   				\item Chemotherapy
    			\item Surgery
					\item \textbf{Radiation therapy}

   			\end{itemize}
				\qquad Energy $\sim$ MeV

   		\column{.5\textwidth}

			\begin{figure}[tb]
    			\centering
    			\includegraphics[width=5cm]{gfx/direct.png}
    			\label{fig:direct}
			\end{figure}

    \end{columns}

	\end{frame}




	\begin{frame}{Radiosensitization of cancer cells with \\ gold nanoparticles (GNP) \footnotesize E $\sim$ keV}
	\footlineextra{Source: https://en.wikibooks.org/wiki/Basic_Physics_of_Digital_Radiography/The_Patient}
	%\centering
	\vspace{.3cm}
	\begin{columns}
    \centering
		\column{0.35\textwidth}

			Photoelectric absorption
			%\vspace{-.8cm}
				\begin{figure}[tb]
						\centering
    				\includegraphics[width=3.5cm]{gfx/photoel2.png}
    				\label{fig:photoel}
				\end{figure}
				%\vspace{-.4cm}
		\column{0.35\textwidth}
	%\centering
			Compton effect
			%\vspace{-.8cm}
				\begin{figure}[tb]
    				\centering
    				\includegraphics[width=3.5cm]{gfx/compton2.png}
    				\label{fig:compton}
				\end{figure}
		\end{columns}
	\end{frame}

	\begin{frame}{Targeting of the GNP to the tumor}
	\begin{columns}[c]
	\column{0.5\textwidth}
	\centering
	\textbf{Passive targeting} \\
	 PEG coating
		\begin{figure}[tb]
    		\centering
    			\includegraphics[width=0.8\textwidth]{gfx/ptargeting.png}
    			\label{fig:ptargeting}
		\end{figure}
	\column{0.5\textwidth}
	\centering
	 \textbf{Active targeting} \\
	 Antibodies
	\begin{figure}[tb]
    			\centering
    			\includegraphics[width=0.8\textwidth]{gfx/atargeting.png}
    			\label{fig:atargeting}
			\end{figure}
	\end{columns}

	\end{frame}

\begin{frame}
\footlineextra{Source: A. Chaudhary , A. Gupta , S. Khan and C. Kanti Nandi \emph{Phys. Chem. Chem. Phys.}, 2014 \textbf{16}}

\begin{columns}[c]


	\column{.3\textwidth}
	%\tableofcontents
  \begin{enumerate}
    \item Synthesis
    \item Characterization
    \item Radiosensitization
  \end{enumerate}
	\column{.6\textwidth}

\begin{figure}[tb]
	\centering
	\includegraphics[width=\textwidth]{gfx/Inhoud}
\end{figure}
	\end{columns}

\end{frame}
%------------------------------------------------
%------------------------------------------------
\section{Synthesis}
%------------------------------------------------
%------------------------------------------------

%------------------------------------------------
\subsection{Chemical protocol}
%------------------------------------------------

	\begin{frame}{Reduction of gold ions to form GNP}

		\begin{figure}[!h]
			\centering
			\includegraphics[width=0.8\textwidth]{gfx/GNPReduction}
		\end{figure}
    \vspace{-2.2cm}
    \Large
      \begin{center}
          Reducing \\agent
      \end{center}

		\vspace{1cm}
\normalsize
		\begin{columns}[c]
		\column{.5\textwidth}

		Gold ions: HAuCl$_4$ solution\\
		Reducing agent: Na$_3$C$_6$H$_5$O$_7$
		\column{.15\textwidth}
		\end{columns}

    \end{frame}

	\begin{frame}[t]
	\frametitle{Reduction of gold ions to form GNP}
  \begin{columns}[c]
    	\column{.5\textwidth}
      \vspace{0.5cm}
      OH$^-\qquad $ C$_6$H$_5$O$_7^-$
      	\column{.5\textwidth
        bla
  \end{columns}

		\vspace{0.5cm}
		\begin{figure}[h!]
			\centering
			\includegraphics[width=0.8\textwidth]{gfx/Reduction}
		\end{figure}
		\vspace{0.5cm}
		\centering
		$6$Au$^3^+$ $+$ C$_6$H$_5$O$_7^{3^-}$ $+$ $15$OH$^-\rightarrow$ Au $+$ $6$CO$_2$ $+$ $10$H$_2$O

	\end{frame}
%------------------------------------------------
\subsection{Size GNP}
%------------------------------------------------
	\begin{frame}[t]
	\frametitle{The amount of citrate controls the size}

		\begin{figure}[tb]
    		\centering
    		\includegraphics[width=0.8\textwidth]{gfx/SizeGNP}
    	\end{figure}

    	\begin{columns}[c]
		\column{.5\textwidth}
			Citrate $1$\%\\
    		\small $100$ml HAuCl$_4$ $0.01$\%
		\column{.15\textwidth}
		\end{columns}
\end{frame}

%------------------------------------------------
%------------------------------------------------

\subsection{Functionalization}
\begin{frame}[t]
\frametitle{PEG for targeting and stabilization}

    \begin{figure}[tb]
    	\centering
    	\includegraphics[width=0.7\textwidth]{gfx/PEG}
    \end{figure}
    \vspace{-1cm}
    \begin{columns}
    	\column{.5\textwidth}
    	\begin{figure}[tb]
    		\centering
    		\includegraphics[width=0.7\textwidth]{gfx/PEG2}
    	\end{figure}
    	\centering
    	$20$k, $10$k, $5$k, $1$k
    	\column{.5\textwidth}
    	\begin{figure}[tb]
    		\centering
    		\includegraphics[width=0.8\textwidth]{gfx/GNPPEG}
    	\end{figure}

    \end{columns}
\end{frame}
\begin{frame}{UV-Vis spectroscopy}
\begin{columns}
	\column{.3\textwidth}
	\begin{enumerate}
		\item Add PEG
		\item Size GNP
		\item Add NaCl
		\item Size GNP
	\end{enumerate}
	\column{.6\textwidth}
	\begin{figure}[tb]
    	\centering
    	\includegraphics[width=\textwidth]{gfx/uvvis.png}
    \end{figure}
\end{columns}
\vspace{.5cm}
\centering
	bigger size $\rightarrow$ too little PEG\\
	same size $\rightarrow$ enough PEG
\end{frame}

\begin{frame}{Results}
\centering
GNP no PEG
\begin{figure}[tb]
            		\centering
            		\includegraphics[width=12cm]{gfx/vis1.png}
        		\end{figure}
\end{frame}
\begin{frame}{Results}
\centering
15nm GNP 20k PEG for different PEG/GNP
      	\begin{overprint}
    		\only<1>{
        		\begin{figure}[tb]
            		\centering
            		\includegraphics[width=12cm]{gfx/vis22.png}
        		\end{figure}
       	 	}
        	\pause
    		\only<2>{
        		\begin{figure}[tb]
            		\centering
            		\includegraphics[width=12cm]{gfx/vis33.png}
        		\end{figure}
        	}
        	\pause
    		\only<3>{
        		\begin{figure}[tb]
            		\centering
            		\includegraphics[width=12cm]{gfx/vis44.png}
        		\end{figure}
        	}


			\end{overprint}
\end{frame}
\section{Characterization} % (fold)
\begin{frame}[t]\frametitle{Overview}
\footlineextra{Source: A. Chaudhary , A. Gupta , S. Khan and C. Kanti Nandi \emph{Phys. Chem. Chem. Phys.}, 2014 \textbf{16}}
\vspace{.7cm}
  \begin{columns}
    \column{.4\textwidth}
    Introduction\\
    Synthesis GNP\\
    \qquad Chemical Protocol\\
    \qquad Size GNP\\
    Stabilization\\
    Characterization\\
    \qquad Size GNP\\
    \qquad \qquad Chemical Protocol\\
    \qquad \qquad UV-VIS \\
    \qquad \qquad TEM\\
    \qquad Hydrodynamic Radius\\
    \qquad \qquad DLS\\
    \column{.5\textwidth}
    \begin{figure}[tb]
    	\centering
    	\includegraphics[width=\textwidth]{gfx/Inhoud}
    \end{figure}
    \end{columns}


\end{frame}
\begin{frame}[t]
\frametitle{Dynamic light scattering (DLS)}
\vspace{-.4cm}
\begin{columns}
	\column{.5\textwidth}
	\begin{figure}[tb]
		\centering
		\includegraphics[width=.5\textwidth]{gfx/GNPPEG2}
	\end{figure}
	\column{.5\textwidth}
	Hydrodynamic radius ($R_h$)\\
	$\rightarrow$ Rayleigh scattering
\end{columns}
\vspace{-.7cm}
\begin{columns}
\column{.65\textwidth}
\column{.35 \textwidth}
	$$g(\tau)=\frac{\langle I(t)\rangle \langle I(t+\tau)\rangle}{\langle I(t)\rangle^2}$$\\
	$$R_2<R_1$$
\end{columns}
\vspace{-2cm}
\begin{columns}
    \column{.5\textwidth}
    \begin{figure}[tb]
    	\centering
    	\includegraphics[width=\textwidth]{gfx/intensity}
    \end{figure}
    \column{.5\textwidth}
    \begin{figure}[tb]
    	\centering
    	\includegraphics[width=\textwidth]{gfx/correlation}
    \end{figure}
\end{columns}

\end{frame}
\begin{frame}[t]\frametitle{Results}
\centering
\vspace{-.3cm}
\begin{center}
\begin{overprint}
        \only<1>{
        \vspace{-.1cm}
        Functionalisation no PEG
        \vspace{-.1cm}
    \begin{figure}[tb]
    	\centering
    	\includegraphics[width=\textwidth]{gfx/DLSzPEG}
    \end{figure}
}
\pauze
\only<2>{
\vspace{-.1cm}
Functionalisation 20k PEG
        \vspace{-.1cm}
	\begin{figure}[tb]
		\centering
		\includegraphics[width=\textwidth]{gfx/DLSmPEG}
	\end{figure}
}
\end{overprint}
\end{center}
\end{frame}

\begin{frame}[t]\frametitle{Results}
    \centering\\
Functionalization $15$nm $20$k PEG

    \begin{center}
	\begin{overprint}
        \only<1>{
    \begin{table}[tb]
    	\centering

    	\begin{tabular}{c|c}
    	\hline

    	\hline
    	\textbf{Proportion} & \textbf{Average} \\
    	\footnotesize{(PEG/GNP)} & \\
    	\hline
    		$5/10$ & $51.93 \pm 2.76$ \\
    		$6/10$ & $80.89 \pm 14.64$\\
    		$7/10$ & $65.24 \pm 14.32$ \\
    		$8/10$ & $83.91 \pm 18.42$\\
    		$9/10$ &  \\
    	\hline

    	\hline
    	\end{tabular}
    \end{table}
    \vspace{.4cm}
        Original functionalization $20$k ($8/10$)
        \vspace{-.5cm}
    \begin{figure}[tb]
    	\centering
    	\includegraphics[width=0.6\textwidth]{gfx/DLS}
    \end{figure}

    }
    \pauze
    \only<2>{
    \begin{table}[tb]
    	\centering

    	\begin{tabular}{c|c||c}
    	\hline

    	\hline
    	\textbf{Proportion} & \textbf{Average} & \textbf{Average}\\
    	\footnotesize{(PEG/GNP)} & & \footnotesize{(centrifuge)}\\
    	\hline
    		$5/10$ & $51.93 \pm 2.76$ & $68.70 \pm 7.99$ \\
    		$6/10$ & $80.89 \pm 14.64$ & $65.16 \pm 11.61$\\
    		$7/10$ & $65.24 \pm 14.32$ & $57.73 \pm 7.72$\\
    		$8/10$ & $83.91 \pm 18.42$ & $72.36 \pm 10.44$\\
    		$9/10$ & & $56.54 \pm 3.91$\\
    	\hline

    	\hline
    	\end{tabular}
    \end{table}
    \vspace{.4cm}
        Original functionalization $20$k ($8/10$)
        \vspace{-.5cm}
    \begin{figure}[tb]
    	\centering
    	\includegraphics[width=0.6\textwidth]{gfx/DLS}
    \end{figure}
    }
    \end{overprint}
    \end{center}

\end{frame}

\begin{frame}{Conclusion}
\vspace{1cm}
\begin{columns}
\column{0.5\textwidth}
\begin{itemize}
    \item Synthesis of GNP
    \item Characterization
    \item Stabilization with neutral PEG
\end{itemize}
\column{0.5\textwidth}
\begin{itemize}
    \item Stabilization with positively charged PEG
    \item X-Rays

    \item Analyze effect on DNA
		\item Solve problem with DLS
\end{itemize}
\end{columns}
\vspace{-2cm}
\begin{figure}[tb]
	\centering
	\includegraphics[width=0.8\textwidth]{gfx/overview}

\end{figure}
\end{frame}

% section characterization (end)
% \begin{frame}{Referenties}


% \begin{thebibliography}{99} % Beamer does not support BibTeX so references must be inserted manually as below
% \setbeamertemplate{bibliography item}[article]
% \bibitem[S. R. Makhsin et all.]{R1} (2012) The effects of size and synthesis methods of gold nanoparticle-conjugated MαHIgG4 for use in an immunochromatographic strip test to detect brugian filariasis
% \newblock  \emph{S. R. Makhsin, K. A. Razak, R. Noordin, N. D. Zakaria and T. S. Chun}, 2012 November 19
% \newblock Universiti Sains Malaysia
% \end{thebibliography}

% \end{frame}

\end{document}
