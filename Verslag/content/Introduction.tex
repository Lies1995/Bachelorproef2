\section{Introduction}
Each year 8.2 million people die from cancer, that is an estimated 13$\%$ of all deaths worldwide. Moreover the number of new cases is expected to increase by 70$\%$ over the next two decades \cite{WHO}. These numbers are a clear indication of the importance of cancer research, which includes the development of new treatments and the enhancement of existing treatments.\\
Cancer is a general term for a large group of diseases that are all characterised by a rapid creation of abnormal cells that grow beyond their usual boundaries. The cells grow out of control and thereby form tumors, because the part of the DNA responsible for cell death is disturbed.\\
The three major cancer treatments that exist today are chemotherapy, surgery and radiation therapy. In chemotherapy anti-cancer drugs are used to damage the cancer cells. These drugs only affect rapid dividing cells, which is a typical characteristic of cancer cells. Cancer can also be treated by removing the tumor by surgery. A third important cancer treatment is radiation therapy, where ionizing radiation is used to damage the DNA of the cancer cells. Because in most cases combinations of these treatments are used, it is estimated that 50$\%$ of all cancer patients undergo radiation therapy as part of their treatment \cite{imaginis}.\\
The problem with ionizing radiation is, that it does not discriminate between malignant and normal tissues. This means that normal tissue is also irradiated. Therefore radiation therapy has an unintentional toxic effect on the healthy tissue surrounding the tumor. These effects range from those that cause mild discomfort to others that are life-threatening. The radiation dose has to be carefully chosen: it has to be high enough to bring damage to the cancer cells, but it cannot be too high because of the negative effect on the healthy tissue.\\
The toxic effect of radiation therapy can be reduced in different ways. For example instead of delivering the total dose of radiation at once, the dose can be fractionated in smaller doses that are delivered over a longer period of time. Dose fractionations offers the opportunity for healthy cells to recover from the previous dose before the next dose is delivered \cite{radiobiology}. Another technique to reduce the effect on healthy tissue is radiosensitization. With this technique it is possible to make the cancer cells more sensitive to ionizing radiation, so that lower radiation doses can be used. This reduces the effect on the surrounding tissue. Here radiosensitization using gold nanoparticles (GNP) is discussed. The absorption of photons is higher for elements with high mass numbers. Therefore when the GNP can be brought inside the tumor, because of the high mass number of gold (Z=79), the ionizing radiation will mainly interact with the nanoparticles. This causes local secondary radiation, which delivers its energy locally, so inside the tumor.\cite{targeting} The reason gold is used instead of other element with high atomic mass, is that gold is biocompatible, which makes it suitable for medical treatment \cite{biocompatible}. \\
The first experimental evidence of the use of GNP to enhance radiation therapy was provided by Hainfeld et al. \cite{expevidence}. Mice with cancer tumors were injected with 1.9 nm diameter GNP and then radiated with 250 kVp X-rays. The combination of GNP and radiation resulted in a one-year survival of 86$\%$ compared to 20$\%$ with radiation therapy alone. Other experiments showed similar evidence of the radiosensitizing effect of GNP \cite{expevidence2}\cite{expevidence1}. These results provide a motivation for further research within the field of nanoparticle enhanced radiation therapy.\\
The main goal of this project is to synthesize GNP of different sizes and to characterize them using different methods. Next the GNP are functionalized with a polyethylene glycol (PEG) coating, which increases the probability of delivering the nanoparticles to the cancer cells. The PEG coating also provides stability to the GNP solution, i.e., prevents them from aggregating. Finally a mixture of DNA and GNP is irradiated and the effect on the DNA is analysed.\\
%#TODO effect PEG omdraaien? we voegen het to voor de targetting en merken dat het een positief effect heeft op de stabilizatie?
\newpage
\section{Theoretical background}
\subsection{Radiation physics}
The ionizing radiation type used for radiation therapy can vary ranging from photons and electrons to protons, neutrons and low-mass ions, but photons are by far the most common form of radiation used in cancer treatment \cite{radtype}. There are three possible ways photons can interact with matter: photoelectric absorption, Compton scattering and pair production. An illustration of these processes is given in figure \ref{fig:inttype}.
\begin{figure}
    \centering
    \begin{subfigure}[h]{0.5\textwidth}
        \includegraphics[width=\textwidth]{gfx/compton}
        \caption{Compton effect}
        \label{fig:gull}
    \end{subfigure}
    ~ %add desired spacing between images, e. g. ~, \quad, \qquad, \hfill etc. 
      %(or a blank line to force the subfigure onto a new line)
    \begin{subfigure}[h]{0.5\textwidth}
        \includegraphics[width=\textwidth]{gfx/photoel}
        \caption{Photoelectric effect}
        \label{fig:tiger}
    \end{subfigure}
    ~ %add desired spacing between images, e. g. ~, \quad, \qquad, \hfill etc. 
    %(or a blank line to force the subfigure onto a new line)
    \begin{subfigure}[h]{0.4\textwidth}
        \includegraphics[width=\textwidth]{gfx/pair_production}
        \caption{Pair production}
        \label{fig:mouse}
    \end{subfigure}
    \caption{Possible interaction types for photons.}\label{fig:inttypes}
		\label{fig:inttype}
\end{figure}\\
In the photoelectric effect, the energy of an incoming photon is transferred to an electron, which is then ejected. The vacancy left by this electron is then filled with another electron from a higher shell. This electron then gives of its excess energy as an characteristic X-ray photon. In some cases the excess energy may be transferred to an outer-shell electron. As a consequence this electron is ejected and is called an Auger electron. The cross section for photoelectric absorption $\tau$ increases for increasing mass number $Z$ and in decreases sharply with the photon energy $E_{\gamma}$:
\begin{equation}
\tau \cong C^{te} \cdot \frac{Z^n}{E_{\gamma}{^{3.5}}}
\end{equation}
with $n$ varying between $4$ and $5$ \cite{Knoll}.\\
The Compton effect is an inelastic scattering between a photon and an electron, where part of the energy of the incoming photon is transferred to the recoiling electron. The cross section for Compton scattering grows linearly with $Z$ and falls of gradually with increasing energy \cite{Knoll}.\\
With pair production, a photon creates an electron-positron pair. The cross section for this process varies approximately with $Z^2$ and increases for increasing energy \cite{Knoll}.\\
Since the cross sections for each of these processes increases with increasing $Z$, it is clear that gold, with its high atomic mass, is suitable for radiosensitization.\\
Which interaction will take place depends on the photon energy and on the atomic number of the absorber, this is illustrated in figure \ref{fig:dominant}. 
\begin{figure}
\begin{center}
 \includegraphics[width=0.7\textwidth]{gfx/dominant}
\end{center}
        \caption{The region of dominance of the different interaction types in function of the atomic number of the absorber and the energy of the incoming photon.}
				\label{fig:dominant}
\end{figure}\\
For the purpose of radiosensitization, local energy delivery is needed. Therefore short range secondary radiation is preferred. That is why the photoelectric effect, which results in a secondary electron, is the preferred interaction. This interaction is dominant in the keV range, so that keV photons will give rise to the best local energy deposition \cite{Leung}. \\
For deeper lying tumors however, low keV radiation is not sufficient because of the limited penetration depth. This is illustrated in figure \ref{fig:atlength}, where the attenuation length is plotted in function of the photon energy for water. The attenuation length is the depth at which the intensity of the radiation is decreased by a factor $1/e$ \cite{lennaert}. 
\begin{figure}
\begin{center}
 \includegraphics[width=0.5\textwidth]{gfx/atlength}
\end{center}
        \caption{The attenuation length of X-rays in water as a function of incomminng photon enery.}
				\label{fig:atlength}
\end{figure}\\
In figure \ref{fig:atlength} it is shown that high keV or MeV radiation is needed to reach deep lying tumors. In this energy region the dominant effect that will occur is Compton scattering (see figure \ref{fig:dominant}), both in tissue and in gold. The Compton effect results in scattered photons and electrons with a broad energy distribution. The energies of many of these photons fall within the region where interaction with gold is more probable than interaction with tissue, so that short-range secondary electrons are produced in large amounts by electromagnetic radiation in any energy range \cite{34}. Therefor even for treatment of deeper lying tumors the radiosensitization effect is present, but it is still preferable to bring the radiation source as close as possible to the tumor.

\subsection{Biological effects}

Soon after the discovery of radioactivity, it became clear that the use of ionizing radiation is not harmless. Generally it is assumed that the negative effects due to radiation exposure are mainly a result of DNA damage. Damage to other parts of the cell, such as membrane damage and cytoplasmic damage do not contribute significantly \cite{41}.\\ Deoxyribonucleic acid or DNA is a molecule that carries the genetic instructions for the functioning of all living organisms and many viruses. It is composed of two polynucleotide strands that form a double helix. Each of these strand stores the same biological information \cite{celbio}. \\
\begin{wrapfigure}{l}{0.4\textwidth}
\begin{center}
\vspace{-20pt}
 \includegraphics[width=0.4\textwidth]{gfx/direct}
\end{center}
        \caption{Direct and indirect effect of ionizing radiation on DNA.}
				\label{fig:direct}
				\vspace{-20pt}
\end{wrapfigure}
When ionizing radiation interacts with biological matter, it causes ionization and excitation of atoms and molecules inside the cells, which can break chemical bonds. Ionizing radiation leads to DNA damage in two different ways. When ions are formed directly in the DNA molecule, the DNA is damaged and one speaks of a direct effect. In the aqueous environment inside the cells, the ionizing radiation can also interact with the water molecules surrounding the DNA. The water molecules are dissociated and free radicals are formed. These are reactive molecules with an unpaired electron that can interact with the DNA causing damage. The radicals that are formed after the dissociation of water are the OH$\cdot$ and H$\cdot$ radical, the dot here indicates the unpaired electron. This is the indirect effect. Both the direct and indirect effect are illustrated in figure \ref{fig:direct}. There are three common types of damage to the DNA due to ionizing radiation: base excision, single strand breaks and double strand breaks. Repair mechanisms exist in the cell for each of these types of damage. \\


\subsection{Targeting}

In order to have a beneficial effect of GNP in radiation therapy, it is important to bring the nanoparticles as close as possible to the DNA of the cancer cells. The uptake of GNP into the nucleus of the cells is only possible below a certain upper size limit \cite{Ken}.\\  Therefore the size of the nanoparticle is a very important parameter.
GNP are known to passively accumulate in cancer cells because of the enhanced permeability and retention (EPR) effect. Because cancer cells are rapid growing cells, tumors have leaky, immature vasculature, so that their blood vessels are more permeable \cite{Jain}. \\  This effect can be enhanced by functionalizing the naked gold nanoparticles with PEG. This PEG coating sterically hinders nonspecific binding of proteins to the surface of the particle and delays the recognition of the particles by the reticuloendothelial system. This increase the circulation time of the GNP in the blood and as a result increases the probability of delivering the nanoparticles to the tumor \cite{Ken}. \\  The hydrodynamic radius of the particles increases because of this PEG coating, therefore the optimal size (of the naked GNP) for uptake is smaller when the particle is functionalized. Finally the functionalization also has a positive effect on the stability of the GNP solution.\\
Besides PEGylation, the GNP can also be coated with antibodies, which actively bind to receptors that are specific for cancer cells. An example of a receptor that is overexpressed in tumors is the epidermal growth factor receptor. Nanoparticles coated with an antibody that corresponds to this receptor are guided to the tumor and bind on its surface \cite{Tiwari}.

% Bronnen
% WHO http://www.who.int/mediacentre/factsheets/fs297/en/
% imaginis http://www.imaginis.com/radiotherapy/cancer-treatment-with-radiation-therapy
% radiobiology http://citeseerx.ist.psu.edu/viewdoc/download?doi=10.1.1.462.2846&rep=rep1&type=pdf
% biocompatible http://www.ncbi.nlm.nih.gov/pubmed/16262332
% targeting: circulation time http://www.ncbi.nlm.nih.gov/pmc/articles/PMC3473940/
% first experimental evidence (proxy) expevidence http://iopscience.iop.org/article/10.1088/0031-9155/49/18/N03/meta
