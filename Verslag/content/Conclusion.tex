\section{Conclusion}
Synthesizing gold nanoparticles following the Turkevich method is a reliable way to obtain particles of a specific size. Though, one always has to keep in mind that smaller particles are also created and that the system is very sensitive to slight deviations of concentration. To be certain about the size distributions of the particles in a colloid, TEM image analysis is the best and most reliable way. In the $15$nm particle colloid a subpopulation has been found, probably due to unequal distribution of the citrate ions in solution. \\
Throughout all measurements the particles functionalized with $1$k PEG provide strange results. First, there seems to be a maximum PEG/GNP proportion at which point the colloid becomes unstable again and the particles start to aggregate. One would expect that once the whole surface of the particles is coated with a PEG-layer, which provides the necessary steric repulsion to prevent aggregation, adding more PEG does not influence this equilibrium. Secondly, the hydrodynamic radius of the $1$k PEG functionalized particles appears to be larger then the $R_h$ of particles with the same core diameter but functionalized with larger PEG. The $\zeta$-potential results are also questionable and vary from extremely negative to even positive. In the gelelectrophoresis some samples functionalized with $1$k appear to be strongly negatively charged. This could indicate that there is too little neutral PEG and the particles are still functionalized with negative citrate ions.
If the $1$k PEG needs to be used for medical applications further research is needed to account for and eliminate this strange behavior.
\\The characterization results for the other samples fulfill our expectations. In measuring the hydrodynamic radius using the DLS technique though, some technical problems were observed with the laser. This can account for unexpected results and large errors. The measurements have to be interpreted carefully before they are generalized. When ignoring the results for particles functionalized with $1$k PEG, the expected trend that functionalization with larger PEG corresponds to larger hydrodynamic radii is observed. \\
The $\zeta$-potential for the functionalized samples is in general measured to be neutral as it should be due to the methoxy end-group of the PEG. \\
When analyzing the DNA damage after the irradiation it is observed that in general the larger particles have a slightly better effect than smaller particles. This however, as stated before, contradicts the literature \cite{lies}. It is expected that this apparent better effect for larger particles is due to a smaller total gold concentration in the samples with smaller particles resulting from the centrifuging process. Furthermore the particles functionalized with the $5$k PEG (the smallest PEG when ignoring the strange results for $1$k PEG) provide the highest damage. The effect of higher dose should be an increase in damage and this is not consequently observed. Most likely this is due to incorrect storing during the time between the preparation of the samples, irradiation and the gelelectrophoresis. This unfortunate timing of the experiments occurred due to technical problems and should be avoided in further experiments.
%#TODO something about the DNA damage due to the others
