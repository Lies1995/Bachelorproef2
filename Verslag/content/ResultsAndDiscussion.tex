\section{Results and Discussion}
\subsection{Core Diameter}
\begin{wraptable}{r}{0.4\textwidth}
  \caption{Core diameter of the particles observed in the TEM images. Exp. Size is the diameter expected from the chemical protocol used}
  \label{tab:TEM}
  \begin{tabular}{ c | c }
    Exp. Size (nm) & Size (nm)\\
    \toprule
    $15$ & $12.98 \pm 0.23$\\
     & $2.99 \pm 0.16$\\
    $30$ & $18.29 \pm 0.23$\\
    $45$ & $46.75 \pm 0.47$
  \end{tabular}
\end{wraptable}
As described in section \ref{subsubsec:CoreDia} TEM images of the particles are created to determine their shape and core diameter. According to the synthesis protocol used (see section \ref{subsubsec:Turk} and appendix) we expect spherical particles.  Several images were created from the three different colloids (expected size $15$nm, $30$nm and $45$nm) in advance of the PEG functionalization, thus naked gold nanoparticles. In figure \ref{fig:TEM} one image of each colloid is shown in combination with a histogram indicating the size distribution. The data in the histogram is a combination of data from several images.\\
The TEM images show approximately spherical nanoparticles. They cluster together due to Van der Waals interactions but do not aggregate due to the electrostatic repulsion of the citrate ions surrounding the naked particles.
The average size calculated from the data is presented in table \ref{tab:TEM}.
For the particles with expected size of $45$nm, the observed average size is slightly bigger and for the smallest particles slightly smaller.%#TODO kleine uitleg waarom de afwijking en waarom we dit wel oke vinden
The particles requiring attention are  the ones with expected size $30$nm. They are more then $10$nm smaller compared to the size the protocol predicts. The reason for this deviation lies most certainly in the experimental handling of the protocol. (Many other experimenters have used the protocol and obtained the expected results) Since the particles are smaller, presumably a higher concentration of citrate is added to the gold solution. This can happen due to wrongful pipetting work or evaporation of the gold solution while bringing it to boiling temperature. The last one seems unlikely since during the heating a condenser was used with a continuous flow of cold water.\\
The size distributions from the two biggest particles, figures \ref{fig:TEM}(b) and (d), indicate a wide spreading around the average.
If more data had been collected a gaussian size distribution would have been expected since the growth of nanoparticles depends various random factors. For the particles of expected size $30$nm the size distribution approximates a normal gaussian distribution whereas the distribution for the biggest particles appear to be negatively skewed. For the smallest particles though, clearly a subpopulation of diameter around $4$nm is present. The reason for this subpopulation is not sure, to little particles were analyzed to draw immediate conclusions. A possibility is that there was an unequal distribution of the citrate in the gold solution due to unsystematic stirring.
\newpage

\begin{figure}[h!]
  \includegraphics[width=\textwidth]{gfx/TEM_res}
  \caption{}
   \label{fig:TEM}
\end{figure}

\newpage

\subsection{UV-Vis}
\begin{figure}
  \includegraphics[width=\textwidth]{gfx/UVVIS_alle}
  \caption{Absorption spectra for naked GNP of expected size $15$nm, $30$nm and $45$nm}
  \label{fig:UV-VIS all naked}
\end{figure}
Figure \ref{fig:UV-VIS all naked} presents the absorption spectra of the naked gold nanoparticles, thus before the functionalization with PEG. The figure clearly indicates the mechanism of the UV-VIS measurements. Bigger particles have a lower surface plasmon frequency hence a bigger wavelength at which maximum absorption occurs.\\
The most important application of the UV-VIS measurements is to determine the necessary proportion of PEG/GNP solution as described in section \ref{subsubsec:RelSize}.
The reasoning to determine the ideal PEG/GNP proportion from the measurements will be explained here for the $20$k PEG. The same reasoning applies for the other PEG's. In figure \ref{fig:UV-VIS 15nm 20k} four absorption spectra (with different PEG/GNP ratio) for the smallest particles ($15$nm) functionalized with $20$k PEG are presented. Both the curves before and after the addition of NaCl (to stimulate aggregation) are shown. The results for the other particles can be found in the appendix. Each time only four (out of ten) absorption spectra are shown for the sake of clarity of the figure. \\
\begin{wraptable}{r}{0.4\textwidth}
  \caption{Necessary PEG/GNP proportions (volume/volume) for each PEG to create a stable colloid. The proportions are valid for a $1$g/l PEG solution and a GNP solution obtained from the protocol described in the appendix. }
  \label{tab:PEG/GNP}
  \begin{tabular}{ c | c }
    PEG & PEG/GNP \\
    \toprule
    $1$k & $4/100$\\
    $5$k& $4/10$\\
    $10$k & $6/10$\\
    $20$k & $8/10$
  \end{tabular}
\end{wraptable}
When no salt is added, the four absorption maxima from the different spectra should coincide and that is also what had been measured, see figure \ref{fig:UV-VIS 15nm 20k}. Even with insufficient PEG-coating the particles will not immediately aggregate together since they are then still slightly coated with negative citrate ions causing additional electrostatic repulsion. When adding NaCl the electrostatic effects will disappear and only the particles coated with enough PEG to provide the necessary steric repulsion will hold their size. Clearly, a $1/10$ ratio is not enough since with NaCl a secondary peak rises at a higher wavelength indicating the presence of bigger particles. The peak of the $4/10$ absorption spectrum broadens upon addition salt indicating still little aggregation. Almost no difference is observed between the $8/10$ and $10/10$ ratio so the ideal proportion is $8/10$ since we look for the minimum necessary PEG/GNP proportion.
\\The determined PEG/GNP proportions for the four different PEG's are listed in table \ref{tab:PEG/GNP}. Remarkable is the fairly low proportion necessary when working with the smallest PEG (length $1$k). When variating the PEG/GNP proportion between $1/10$ and $10/10$ it was observed that the lower the proportion, the better the reaction of the colloid to the salt. Lowering even further this trend remained, see figure \ref{fig:UV-VIS 15nm 1k}. This is strange and contradicts the theory. It is correct that less PEG (in mass) is needed when using the smaller PEG since in $1$g more PEG molecules are present (This is observed for the other PEG's). However, this doesn't explain why when using $1$k PEG there suddenly seems to be a maximum PEG proportion at which the colloid becomes unstable again. Since this is only observed with the smallest PEG it has to have something to do with the length of the PEG chain. In a shorter PEG chain the negative charge of the thiol group becomes more significant. When there's an over-saturation of PEG in the colloid, the negative charges of the thiol groups at the surface of the particles will start to repel each other. This process causes a weaker bound between the PEG and the nanoparticles facilitating aggregation upon NaCl addition. This explanation has not been tested nor experimentally verified due to time limits. It is worth investigating this strange interaction of GNP and $1$k PEG when a stable GNP colloid with smallest particles possible is desired. 
After the necessary PEG/GNP proportion was determined based on the absorption spectra with the smallest particles, control measurements were done with the absorption spectra from the bigger particles. An example with $20$k PEG is shown in figure \ref{fig:UV-VIS all 20k}. This figure indeed indicates that all three colloids don't aggregate after being mixed with NaCl.

\begin{figure}[h!]
	\centering
	\includegraphics[width=\textwidth]{gfx/UVVIS_20k}
	\caption{Absorption spectra for 15nm GNP (expected size) functionalized with 20k PEG for different PEG/GNP proportions, with and without NaCl.}
  \label{fig:UV-VIS 15nm 20k}
\end{figure}

\begin{figure}[h!]
	\centering
	\includegraphics[width=\textwidth]{gfx/UVVIS_alle_20k}
	\caption{Optical density in function of wavelength for 15, 30 and 45nm GNP, with 20k PEG with PEG/GNP=8/10.}
  \label{fig:UV-VIS all 20k}

\end{figure}

\subsection{Zeta potential}
\begin{table}[h!]
  \centering
  \caption{$\zeta$-potential measurement results in mV for three different particles, with and without functionalisation using PEG.}
\begin{tabular}{c||c|c|c|c|c}
  Size (nm) & No PEG  & $1$k PEG & $5$k PEG & $10$k PEG & $20$k PEG\\
    \hline
  \hline
  $15$&$-33.73\pm1.85$&$-13.08\pm1.67$ &$-5.20\pm1.46 $&$-5.87 \pm1.65$&$-4.45\pm1.28$ \\
  $30$&$-20.65 \pm 1.89$ & $4.48 \pm 1.46$ & $0.76 \pm 1.20$ & $1.06 \pm 1.10$ & $-10.62\pm2.09$\\
  $45$&$-24.70 \pm 3.57$ &$-9.05\pm1.32$ & $1.77 \pm 1.60$ & $-3.51 \pm 2.14$ & $-7.90 \pm 1.69$ \\

\end{tabular}
\end{table}

\begin{figure}
  \centering
  \begin{subfigure}[t]{\textwidth}
    \includegraphics[width=\textwidth]{gfx/Zeta_45nm}
    \caption{}
    \label{fig:Zeta_45}
  \end{subfigure}
  \begin{subfigure}[t]{\textwidth}
    \includegraphics[width=\textwidth]{gfx/Zeta_45nm_fit.png}
    \caption{}
    \label{fig:Zeta_45_fit}
  \end{subfigure}
  \caption{$\zeta$-potential measurement results for particles of 45nm without funtionalisation (a) data and (b) Lorentzian fit. Ten measurements were done and the average was calculated (black curve).}
\end{figure}

\subsection{Hydrodynamic Radius (DLS)}
\begin{table}[h!]
  \centering
  \caption{DLS measurement results in nm for three different particles, of expected size $15$nm, $30$nm and $45$nm, with and without functionalisation using PEG. The length of the PEG chain varies from 20k to 1k .}
\begin{tabular}{c || c | c | c | c | c }
  Size (nm) & No PEG  & $1$k PEG & $5$k PEG & $10$k PEG & $20$k PEG\\
  \hline
  \hline
  $15$ &$28.37 \pm 1.53$  & $135.72 \pm 38.01$ & $40.26\pm 1.25$ & $54.03 \pm 0.56$ & $66.00 \pm 4.71$\\
  & $9.70\pm 1.44$& & &$14.05 \pm 4.13$ \\
  $30$ & $30.03 \pm 1.25$ & $99.61 \pm 17.33$ & $45.41 \pm 1.26$ & $53.28 \pm 1.66$ & $61.29 \pm 1.50$\\
  & $4.99 \pm 1.55$ & $25.37 \pm 1.84$ & $7.85 \pm 1.05$ & $5.90 \pm 0.97$ & $7.36 \pm 1.73$\\
  $45$ & $50.75 \pm 2.05$ & $109.54 \pm 0.69$ & $70.04 \pm 3.06$ & $71.56 \pm 1.87$ & $76.14 \pm 0.66$ \\
  & $3.28 \pm 0.79$ & & $10.67 \pm 1.44$ & $10.74 \pm 1.13$ & $5.15 \pm 0.74$ \\

\end{tabular}
\end{table}
\begin{figure}[h!]
  \centering
  \includegraphics[width=0.8\textwidth]{gfx/DlsALl.png}
  \caption{DLS measurement results for three different particles, of expected size $15$nm, $30$nm and $45$nm, before (transparent) and after PEG functionalisation.}
\end{figure}
