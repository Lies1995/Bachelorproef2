\section{Results and Discussion}
\subsection{Core Diameter (TEM)}
\begin{wraptable}{r}{0.4\textwidth}
  \caption{Core diameter of the particles observed in the TEM images. Exp. Size is the diameter expected from the chemical protocol used}
  \label{tab:TEM}
  \begin{center}
  \begin{tabular}{ c | c }
    Exp. Size (nm) & Size (nm)\\
    \toprule
    $15$ & $12.98 \pm 0.23$\\
     & $2.99 \pm 0.16$\\
    $30$ & $18.29 \pm 0.23$\\
    $45$ & $46.75 \pm 0.47$
  \end{tabular}
  \end{center}
\end{wraptable}
As described in section \ref{subsubsec:CoreDia} TEM images of the particles are created to determine their shape and core diameter. According to the synthesis protocol used (see section \ref{subsubsec:Turk} and appendix) we expect spherical particles.  Several images were created from the three different colloids (expected size $15$nm, $30$nm and $45$nm) in advance of the PEG functionalization, thus naked gold nanoparticles. In figure \ref{fig:TEM} one image of each colloid is shown in combination with a histogram indicating the size distribution. The data in the histogram is a combination of data from several images.\\
The TEM images show approximately spherical nanoparticles. They cluster together due to Van der Waals interactions but do not aggregate due to the electrostatic repulsion of the citrate ions surrounding the naked particles.
The average size calculated from the data is presented in table \ref{tab:TEM}.
For the particles with expected size of $45$nm, the observed average size is slightly bigger and for the smallest particles slightly smaller. This diversion though was expected due to the limitations on the work precision.
The particles requiring attention are  the ones with expected size $30$nm. They are more then $10$nm smaller compared to the size the protocol predicts. The reason for this deviation lies most certainly in the experimental handling of the protocol. (Many other experimenters have used the protocol and obtained the expected results) Since the particles are smaller, presumably a higher concentration of citrate is added to the gold solution. This can happen due to wrongful pipetting work or evaporation of the gold solution while bringing it to boiling temperature. The last one seems unlikely since during the heating a condenser was used with a continuous flow of cold water.\\
The size distributions from the two biggest particles, figures \ref{fig:TEM}(b) and (d), indicate a wide spreading around the average.
If more data had been collected a gaussian size distribution would have been expected since the growth of nanoparticles depends various random factors. For the particles of expected size $30$nm the size distribution approximates a normal gaussian distribution whereas the distribution for the biggest particles appear to be negatively skewed. For the smallest particles though, clearly a subpopulation of diameter around $4$nm is present. The reason for this subpopulation is not sure, to little particles were analyzed to draw immediate conclusions. A possibility is that there was an unequal distribution of the citrate in the gold solution due to unsystematic stirring.


\begin{figure}[h!]
  \includegraphics[width=\textwidth]{gfx/TEM_res}
  \caption{}
   \label{fig:TEM}
\end{figure}

\subsection{Relative Size (UV-VIS)}
\begin{figure}
  \includegraphics[width=\textwidth]{gfx/UVVIS_alle}
  \caption{Absorption spectra for naked GNP of expected size $15$nm, $30$nm and $45$nm}
  \label{fig:UV-VIS all naked}
\end{figure}
\begin{wraptable}{r}{0.4\textwidth}
  \caption{Necessary PEG/GNP proportions (volume/volume) for each PEG to create a stable colloid. The proportions are valid for a $1$g/l PEG solution and a GNP solution obtained from the protocol described in the appendix. }
  \label{tab:PEG/GNP}
  \begin{center}
  \begin{tabular}{ c | c }
    PEG & PEG/GNP \\
    \toprule
    $1$k & $4/100$\\
    $5$k& $4/10$\\
    $10$k & $6/10$\\
    $20$k & $8/10$
  \end{tabular}
    \end{center}
\end{wraptable}
Figure \ref{fig:UV-VIS all naked} presents the absorption spectra of the naked gold nanoparticles, thus before the functionalization with PEG. The figure clearly indicates the mechanism of the UV-VIS measurements. Bigger particles have a lower surface plasmon frequency hence a bigger wavelength at which maximum absorption occurs.\\
The most important application of the UV-VIS measurements is to determine the necessary proportion of PEG/GNP solution as described in section \ref{subsubsec:RelSize}.
The reasoning to determine the ideal PEG/GNP proportion from the measurements will be explained here for the $20$k PEG. The same reasoning applies for the other PEG's. In figure \ref{fig:UV-VIS 15nm 20k} four absorption spectra (with different PEG/GNP ratio) for the smallest particles ($15$nm) functionalized with $20$k PEG are presented. Both the curves before and after the addition of NaCl (to stimulate aggregation) are shown. The results for the other particles can be found in the appendix. Each time only four (out of ten) absorption spectra are shown for the sake of clarity of the figure. \\

When no salt is added, the four absorption maxima from the different spectra should coincide and that is also what had been measured, see figure \ref{fig:UV-VIS 15nm 20k}. Even with insufficient PEG-coating the particles will not immediately aggregate together since they are then still slightly coated with negative citrate ions causing additional electrostatic repulsion. When adding NaCl the electrostatic effects will disappear and only the particles coated with enough PEG to provide the necessary steric repulsion will hold their size. Clearly, a $1/10$ ratio is not enough since with NaCl a secondary peak rises at a higher wavelength indicating the presence of bigger particles. The peak of the $4/10$ absorption spectrum broadens upon addition salt indicating still little aggregation. Almost no difference is observed between the $8/10$ and $10/10$ ratio so the ideal proportion is $8/10$ since we look for the minimum necessary PEG/GNP proportion.
\\The determined PEG/GNP proportions for the four different PEG's are listed in table \ref{tab:PEG/GNP}. Remarkable is the fairly low proportion necessary when working with the smallest PEG (length $1$k). When variating the PEG/GNP proportion between $1/10$ and $10/10$ it was observed that the lower the proportion, the better the reaction of the colloid to the salt. Lowering even further this trend remained, see figure \ref{fig:UV-VIS 15nm 1k}. This is strange and contradicts the theory. It is correct that less PEG (in mass) is needed when using the smaller PEG since in $1$g more PEG molecules are present (This is observed for the other PEG's). However, this doesn't explain why when using $1$k PEG there suddenly seems to be a maximum PEG proportion at which the colloid becomes unstable again. Since this is only observed with the smallest PEG it has to have something to do with the length of the PEG chain. In a shorter PEG chain the negative charge of the thiol group becomes more significant. When there's an over-saturation of PEG in the colloid, the negative charges of the thiol groups at the surface of the particles will start to repel each other. This process causes a weaker bound between the PEG and the nanoparticles facilitating aggregation upon NaCl addition. This explanation has not been tested nor experimentally verified due to time limits. It is worth investigating this strange interaction of GNP and $1$k PEG when a stable GNP colloid with smallest particles possible is desired.
After the necessary PEG/GNP proportion was determined based on the absorption spectra with the smallest particles, control measurements were done with the absorption spectra from the bigger particles. An example with $20$k PEG is shown in figure \ref{fig:UV-VIS all 20k}. This figure indeed indicates that all three colloids don't aggregate after being mixed with NaCl.

\begin{figure}[h!]
	\centering
	\includegraphics[width=\textwidth]{gfx/UVVIS_20k}
	\caption{Absorption spectra for 15nm GNP (expected size) functionalized with 20k PEG for different PEG/GNP proportions, with and without NaCl.}
  \label{fig:UV-VIS 15nm 20k}
\end{figure}

\begin{figure}[h!]
	\centering
	\includegraphics[width=\textwidth]{gfx/UVVIS_alle_20k}
	\caption{Optical density in function of wavelength for 15, 30 and 45nm GNP, with 20k PEG with PEG/GNP=8/10.}
  \label{fig:UV-VIS all 20k}

\end{figure}
\newpage
\subsection{Hydrodynamic Radius (DLS)}
Figure \ref{fig:DLS no PEG} contains three histograms, one for each particle, showing the detected size by the nanoQ software for all ten measurements. These measurements are made before the functionalization thus represent the hydrodynamic radius of naked particles. For the two biggest particles the measurements indicate subpopulation. These smaller particles were also observed in the TEM analysis although it were especially the smallest particles that had a subpopulation (see figure \ref{fig:TEM}(f)). This subpopulation is not discovered with this detection method perhaps because they are to small.  For each particle the main population has a slightly bigger radius than the expected size. This is normal since the particles are still in an aqueous solution and will be functionalized with citrate ions and water molecules.
Figure \ref{fig:DLS} shows the results for the hydrodynamic radius for all particles and all kinds of PEG's. On this figure for each combination (kind of PEG and size of GNP) only one result is shown. Mostly though the DLS analysis resulted in finding two populations of particles, a polydisperse solution. In table \ref{tab:DLS} the exact numerical results and errors are listed, including the ones for the second (smaller) population.
\begin{wrapfigure}{r}{0.5\textwidth}
  \begin{subfigure}[h]{0.48\textwidth}
    \includegraphics[width=\textwidth]{gfx/8}
  \end{subfigure}
  \\
  \begin{subfigure}[h]{0.48\textwidth}
    \includegraphics[width=\textwidth]{gfx/124}
  \end{subfigure}
  \\
  \begin{subfigure}[h]{0.48\textwidth}
    \includegraphics[width=\textwidth]{gfx/25}
  \end{subfigure}
  \caption{}
  \label{fig:DLS no PEG}
\end{wrapfigure}

In a figure such as figure \ref{fig:DLS} you would expect to see an augmenting trend moving from No PEG to 20k PEG in hydrodynamic radius. Also, the bigger particles should always be bigger than the smaller particles.\\
The first property is violated when using the $1$k PEG. A higher hydrodynamic radius when functionalizing with $1$k PEG than when functionalizing with a bigger PEG $5$k, $10$k and $20$k is strange and seems to be a wrong measurement.
The measurements were repeated several time and extra attention has been paid.
Sometimes it was observed that the power of the incident laser fluctuated abnormally reaching frequently the maximum count rate. Other times the opposite happened.
Even with the highest laser power not enough counts were recorded (the ideal instantaneous count rate is between 2500-3000). These problems affect the correctness of the calculations and could announce for the strange values for $R_h$ and huge error bars on the measurements. \\
The second property is violated when using the $20$k PEG. Here the $15$nm particles suddenly have a bigger hydrodynamic radius than the $30$nm particles. These measurement were also repeated several times an the observed problem was a lack of counts. In trying to solve the problem the sample was centrifuged but this didn't improve anything. Since the results were consistently wrong, new samples from $15$nm particles with variating PEG/GNP proportion (from $5/10$-$9/10$) were analyzed in the hope to find an answer. These samples got also centrifuged since the previous measurements had big errors.  The results are shown in table \ref{tab:15nm 20k var}. Centrifuging the samples to obtain a higher count rate reduces slightly the errors and the average values are more in the line of what is expected. Considering the good results for the not centrifuged and centrifuged samples and relatively small errors, the $5/10$ PEG/GNP proportion seems to be a better choice for the smallest particles when functionalizing with 20k PEG. A control UV-VIS measurement was performed to check whether the colloid with this PEG/GNP proportion is stable and turned out positively. \\
An answer for this strange behavior could be the steric hindrance of the long PEG chains on each other. When packed tightly on the surface the PEG chains will be elongated while with lower proportions the chains can be more compact. This is called the mushroom or brush effect. %#TODO figure
But still, this cannot be the whole story. When synthesizing the particles, all solutions started with an equal amount of gold ions. Hence, in the colloid with the smallest particles a lot more surface needs to be covered with PEG than for the bigger particles (in total). From this all we can state that the concentration of PEG necessary to stabilize the colloid is a complex system and depends on many factors. The details of all interactions taking place in the coating mechanism is an interesting topic to investigate but beyond the scope of this project.\\
%#something about the subpopulation
For this project we conclude first that the functionalization with $1$k PEG behaves strange thus further results with these GNP's have to be interpreted carefully and secondly that the colloid with the smallest GNP ($15$nm) functionalized with the biggest PEG $20$k should have a PEG/GNP proportion of $5/10$ and $8/10$ as was determined with the UV-VIS measurements (see table \ref{tab:PEG/GNP}).
\begin{figure}
  \centering
  \includegraphics[width=0.8\textwidth]{gfx/DlsALl.png}
  \caption{DLS measurement results of the hydrodynamic radius for three different particles, of expected size $15$nm, $30$nm and $45$nm, before (transparent) and after PEG functionalisation.}
  \label{fig:DLS}
\end{figure}
\begin{table}[h!]
  \centering
  \caption{DLS measurement results of the hydrodynamic radius in nm for three different particles, of expected size $15$nm, $30$nm and $45$nm, with and without functionalisation using PEG. The length of the PEG chain varies from 20k to 1k. The used PEG/GNP proportions for each PEG are listed in table \ref{tab:PEG/GNP}.}
  \label{tab:DLS}
\begin{tabular}{c || c | c | c | c | c }
  Size (nm) & No PEG  & $1$k PEG & $5$k PEG & $10$k PEG & $20$k PEG\\
  \hline
  \hline
  $15$ &$28.37 \pm 1.53$  & $135.72 \pm 38.01$ & $40.26\pm 1.25$ & $54.03 \pm 0.56$ & $66.00 \pm 4.71$\\
  & $9.70\pm 1.44$& & &$14.05 \pm 4.13$ \\
  $30$ & $30.03 \pm 1.25$ & $99.61 \pm 17.33$ & $45.41 \pm 1.26$ & $53.28 \pm 1.66$ & $61.29 \pm 1.50$\\
  & $4.99 \pm 1.55$ & $25.37 \pm 1.84$ & $7.85 \pm 1.05$ & $5.90 \pm 0.97$ & $7.36 \pm 1.73$\\
  $45$ & $50.75 \pm 2.05$ & $109.54 \pm 0.69$ & $70.04 \pm 3.06$ & $71.56 \pm 1.87$ & $76.14 \pm 0.66$ \\
  & $3.28 \pm 0.79$ & & $10.67 \pm 1.44$ & $10.74 \pm 1.13$ & $5.15 \pm 0.74$ \\

\end{tabular}
\end{table}
\begin{table}
  \centering
  \caption{DLS measurement results of the hydrodynamic radius in nm for the smallest particles (Exp. size $15$nm) an biggest PEG $20$k for variating PEG/GNP proportion. }
  \label{tab:15nm 20k var}
  \begin{tabular}{c||c |c}
    PEG/GNP & Not centrifuged & centrifuged \\
    \toprule
    $5/10$ & $51.93 \pm 1.38$ & $69.70 \pm 3.02$\\
    $6/10$ & $80.89 \pm 5.18$ & $65.16 \pm 5.80$\\
    $7/10$ & $65.24 \pm 7.16$ & $57.73 \pm 2.73$\\
    $8/10$ & $83.91 \pm 6.14$ & $72.36 \pm 7.38$\\
    $9/10$ & $129.31 \pm 13.74$ & $56.54 \pm 1.38$\\
  \end{tabular}
\end{table}

\subsection{Stability of the colloid ($\zeta$-potential)}
The $\zeta$-potential measurements are done to determine the particle surface charge. Ten measurement are done for each combination of the three different GNP, without PEG and with $1$k, $5$k, $10$k and $20$k PEG. The results of the measurements for the $45$nm particles without PEG are shown in figure \ref{fig:Zeta_45}, it can be seen that there is some fluctuation in these results. The average is calculated and plotted as the black curve in figure \ref{fig:Zeta_45}. These results and their average where then fitted with a Lorentzian fit, as is shown in figure \ref{fig:Zeta_45_fit}. \\
The results of the fits of the averages are shown in table \ref{tab:zeta}, together with a statistical error. It is clearly visible that when no PEG is used, the $\zeta$-potential is significantly more negative, then when PEG is used. This is as expected, because without PEG there is a negative $\zeta$-potential, which originates from the negative citrate molecules. For the smallest GNP, the $\zeta$-potential is approximately $-30$mV, so that this solution can be considered stable. For the larger nanoparticles however, the $\zeta$-potential is slightly larger than $-30$mV (smaller in absolute values). Therefor it is more probable that these solutions are unstable. So it is uncertain whether the electrostatic repulsion is large enough to provide a stable solution, fortunately a PEG coating is added for the targeting, this PEG coating will provide stability due to steric repulsion. As can be seen in table \ref{tab:zeta}, coating the GNP with PEG has a neutralizing effect on the $\zeta$-potential, this is as expected. For some samples this neutralizing effect is much more pronounced and the $\zeta$-potential is approximately zero, for example $30$nm with $5$k PEG. For others the effect is smaller, as can be seen for example for the $15$nm particles. One would expect that larger PEG molecules lead to a less negative $\zeta$-potential, because larger PEG molecules have a larger neutralizing effect. This however is only observed here for the smallest GNP. It is not clear why the results of the larger nanoparticles are inconsistent with the expectations.  Another remarkable result of the $\zeta$-potential measurements is that of the $30$nm GNP with $1$k PEG where a $\zeta$-potential of $4.48$mV is measured. Because of the negatively charged citrate, a negative $\zeta$-potential is expected. This can be neutralized due to PEG, but a positive $\zeta$-potential is unexpected and cannot be explained. Additional $\zeta$-potential measurements can possibly give some answers, but were not done here due to lack of time.
\begin{table}[h!]
  \centering
  \caption{$\zeta$-potential measurement results in mV for three different particles, with and without functionalisation using PEG.}
	\label{tab:zeta}
\begin{tabular}{c||c|c|c|c|c}
  Size (nm) & No PEG  & $1$k PEG & $5$k PEG & $10$k PEG & $20$k PEG\\
    \hline
  \hline
  $15$&$-33.73\pm1.85$&$-13.08\pm1.67$ &$-5.20\pm1.46 $&$-5.87 \pm1.65$&$-4.45\pm1.28$ \\
  $30$&$-20.65 \pm 1.89$ & $4.48 \pm 1.46$ & $0.76 \pm 1.20$ & $1.06 \pm 1.10$ & $-10.62\pm2.09$\\
  $45$&$-24.70 \pm 3.57$ &$-9.05\pm1.32$ & $1.77 \pm 1.60$ & $-3.51 \pm 2.14$ & $-7.90 \pm 1.69$ \\

\end{tabular}
\end{table}

\begin{figure}
  \centering
  \begin{subfigure}[t]{\textwidth}
    \includegraphics[width=\textwidth]{gfx/Zeta_45nm}
    \caption{}
    \label{fig:Zeta_45}
  \end{subfigure}
  \begin{subfigure}[t]{\textwidth}
    \includegraphics[width=\textwidth]{gfx/Zeta_45nm_fit.png}
    \caption{}
    \label{fig:Zeta_45_fit}
  \end{subfigure}
	label{fig:zeta_45}
  \caption{$\zeta$-potential measurement results for particles of 45nm without funtionalisation (a) data and (b) Lorentzian fit. Ten measurements were done and the average was calculated (black curve).}
\end{figure}
