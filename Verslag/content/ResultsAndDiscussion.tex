\section{Results and Discussion}
\subsection{Core Diameter}
\begin{wraptable}{r}{0.4\textwidth}
  \caption{Core diameter of the particles observed in the TEM images. Exp. Size is the diameter expected from the chemical protocol used}
  \label{tab:TEM}
  \begin{tabular}{ c | c }
    Exp. Size (nm) & Size (nm)\\
    \toprule
    $15$ & $12.98 \pm 0.23$\\
     & $2.99 \pm 0.16$\\
    $30$ & $18.29 \pm 0.23$\\
    $45$ & $46.75 \pm 0.47$
  \end{tabular}

\end{wraptable}
As described in section \ref{subsubsec:CoreDia} TEM images of the particles are created to determine their shape and core diameter. According to the synthesis protocol used (see section \ref{subsubsec:Turk} and appendix) we expect spherical particles.  Several images were created from the three different colloids (expected size $15$nm, $30$nm and $45$nm) in advance of the PEG functionalization, thus naked gold nanoparticles. In figure \ref{fig:TEM} one image of each colloid is shown in combination with a histogram indicating the size distribution. The data in the histogram is a combination of data from several images.\\
The TEM images show approximately spherical nanoparticles. They cluster together due to Van der Waals interactions but do not aggregate due to the electrostatic repulsion of the citrate ions surrounding the naked particles.
The average size calculated from the data is presented in table \ref{tab:TEM}.
For the particles with expected size of $45$nm, the observed average size is slightly bigger and for the smallest particles slightly smaller.%#TODO kleine uitleg waarom de afwijking en waarom we dit wel oke vinden
The particles requiring attention are  the ones with expected size $30$nm. They are more then $10$nm smaller compared to the size the protocol predicts. The reason for this deviation lies most certainly in the experimental handling of the protocol. (Many other experimenters have used the protocol and obtained the expected results) Since the particles are smaller, presumably a higher concentration of citrate is added to the gold solution. This can happen due to wrongful pipetting work or evaporation of the gold solution while bringing it to boiling temperature. The last one seems unlikely since during the heating a condenser was used with a continuous flow of cold water.\\
The size distributions from the two biggest particles, figures \ref{fig:TEM}(b) and (d), indicate a wide spreading around the average.
If more data had been collected a gaussian size distribution would have been expected since the growth of nanoparticles depends various random factors. For the particles of expected size $30$nm the size distribution approximates a normal gaussian distribution whereas the distribution for the biggest particles appear to be negatively skewed. For the smallest particles though, clearly a subpopulation of diameter around $4$nm is present. The reason for this subpopulation is not sure, to little particles were analyzed to draw immediate conclusions. A possibility is that there was an unequal distribution of the citrate in the gold solution due to unsystematic stirring.
\begin{figure}[h!]
  \includegraphics[width=\textwidth]{gfx/TEM_res}
  \caption{}
  \label{fig:TEM}
\end{figure}

\subsection{UV-Vis}


\begin{figure}[tb]
	\centering
	\includegraphics[width=\textwidth]{gfx/UVVIS_20k_1}
	\caption{Optical density in function of wavelength for 15nm GNP with 20k PEG for different PEG/GNP proportions, with and without NaCl.}
\end{figure}

\begin{figure}[tb]
	\centering
	\includegraphics[width=\textwidth]{gfx/UVVIS_alle_20k}
	\caption{Optical density in function of wavelength for 15, 30 and 45nm GNP, with 20k PEG with PEG/GNP=8/10.}
\end{figure}

\subsection{Zeta potential}
\begin{table}[h!]
  \centering
  \caption{$\zeta$-potential measurement results in mV for three different particles, with and without functionalisation using PEG.}
\begin{tabular}{c||c|c|c|c|c}
  Size (nm) & No PEG  & $1$k PEG & $5$k PEG & $10$k PEG & $20$k PEG\\
    \hline
  \hline
  $15$&$-33.73\pm1.85$&$-13.08\pm1.67$ &$-5.20\pm1.46 $&$-5.87 \pm1.65$&$-4.45\pm1.28$ \\
  $30$&$-20.65 \pm 1.89$ & $4.48 \pm 1.46$ & $0.76 \pm 1.20$ & $1.06 \pm 1.10$ & $-10.62\pm2.09$\\
  $45$&$-24.70 \pm 3.57$ &$-9.05\pm1.32$ & $1.77 \pm 1.60$ & $-3.51 \pm 2.14$ & $-7.90 \pm 1.69$ \\

\end{tabular}
\end{table}

\begin{figure}
  \centering
  \begin{subfigure}[t]{\textwidth}
    \includegraphics[width=\textwidth]{gfx/Zeta_45nm}
    \caption{}
    \label{fig:Zeta_45}
  \end{subfigure}
  \begin{subfigure}[t]{\textwidth}
    \includegraphics[width=\textwidth]{gfx/Zeta_45nm_fit.png}
    \caption{}
    \label{fig:Zeta_45_fit}
  \end{subfigure}
  \caption{$\zeta$-potential measurement results for particles of 45nm without funtionalisation (a) data and (b) Lorentzian fit. Ten measurements were done and the average was calculated (black curve).}
\end{figure}

\subsection{Hydrodynamic Radius (DLS)}
\begin{table}[h!]
  \centering
  \caption{DLS measurement results in nm for three different particles, of expected size $15$nm, $30$nm and $45$nm, with and without functionalisation using PEG. The length of the PEG chain varies from 20k to 1k .}
\begin{tabular}{c || c | c | c | c | c }
  Size (nm) & No PEG  & $1$k PEG & $5$k PEG & $10$k PEG & $20$k PEG\\
  \hline
  \hline
  $15$ &$28.37 \pm 1.53$  & $135.72 \pm 38.01$ & $40.26\pm 1.25$ & $54.03 \pm 0.56$ & $66.00 \pm 4.71$\\
  & $9.70\pm 1.44$& & &$14.05 \pm 4.13$ \\
  $30$ & $30.03 \pm 1.25$ & $99.61 \pm 17.33$ & $45.41 \pm 1.26$ & $53.28 \pm 1.66$ & $61.29 \pm 1.50$\\
  & $4.99 \pm 1.55$ & $25.37 \pm 1.84$ & $7.85 \pm 1.05$ & $5.90 \pm 0.97$ & $7.36 \pm 1.73$\\
  $45$ & $50.75 \pm 2.05$ & $109.54 \pm 0.69$ & $70.04 \pm 3.06$ & $71.56 \pm 1.87$ & $76.14 \pm 0.66$ \\
  & $3.28 \pm 0.79$ & & $10.67 \pm 1.44$ & $10.74 \pm 1.13$ & $5.15 \pm 0.74$ \\

\end{tabular}
\end{table}
\begin{figure}[h!]
  \centering
  \includegraphics[width=0.8\textwidth]{gfx/DlsALl.png}
  \caption{DLS measurement results for three different particles, of expected size $15$nm, $30$nm and $45$nm, before (transparent) and after PEG functionalisation.}
\end{figure}
