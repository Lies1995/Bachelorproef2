\section{Materials and Methods}
The project consists of three major parts. First the gold nanoparticles are synthesized using the Turkevich method \cite{Turkevich} followed by thorough characterization of the particles using several techniques: TEM, UV-VIS, DLS and $\zeta$-potential. Finally the radiosensitization effect of the GNP is tested by irradiating samples of circular DNA with GNP. \\
This section gives an overview of all techniques and their mechanism used in this project.
\subsection{Synthesis GNP}
\subsubsection{Method of Turkevich}
\label{subsubsec:Turk}
The synthesis of gold nanoparticles is a two step process.  Small seeds of atomic gold are formed after the addition of an reducing agent to an aqueous solution of gold ions, this is the nucleation step. Secondly there is the growth process where small particles aggregate together to form larger particles.

\begin{wrapfigure}{r!}{0.6\textwidth}
  \begin{center}
  \includegraphics[width=0.55\textwidth]{gfx/reduction}
\end{center}
\caption{The synthesis of GNP using the sodiumcitrate (NA$_3$C$_6$H$_5$O$_7$) as an reducing agent.}
\label{fig:reduction}
\end{wrapfigure} 

As mentioned above the Turkevich method is used. Sodiumcitrate (NA$_3$C$_6$H$_5$O$_7$) is added to an tetrachlorauric acid (HAuCl$_4$) aqueous solution to reduce the Au$^{3+}$ ions (see figure \ref{fig:reduction}), the nucleation process has started. After the reduction, the negative citrate ions remain on the surface of the newly synthesized gold seed producing a negative surface potential. At first the electrostatic repulsion is low and the Van der Waals forces cause the seeds to aggregate and form larger particles. During this growth process more and more citrate ions cover the surface of the particles and eventually establish a sufficiently high electrostatic potential to prevent further growth of the nanoparticles. In this manner the sodiumcitrate both starts and ends the synthesis of the nanoparticles and thereby controlling the size of the nanoparticles. Adding more citrate to the tetrachlorauric acid solution will stop the growth process sooner creating smaller nanoparticles.
For this project and the further medical applications it is interesting to do the experiments (functionalization, characterization and radiosensitization) for particles of different sizes. Particles of $15$nm, $30$nm and $45$nm are synthesized following the method described in the appendix.
\subsubsection{Functionalisation with PEG}
\label{subsubsec:Funct}
In order to have an optimal targeting of the GNP to the tumor, the particles are coated with an layer of PEG (Polyethyleen glycol) derivates. One of the end hydroxyl groups is substituted by a sulfhydryl group (SH). It substitution ensures a favorable PEG-GNP bound.
In an aqueous environment this group deprotonates, forming an radical (RS$\cdot$). The PEG$_n$ thiol is now negatively charged. It will substitute the citrate ions on the surface of the GNP since a thiolate-gold bound is comparable in strength to that of the gold-gold bound \cite{gold-surfur}. The second hydroxyl substitution will determine the surface properties of the particles in the colloid. In this project substitution with a methyl-group creating an neutral surface potential is done.
Thus, the GNP are coated with an layer of Methoxy-PEG$_n$ thiol, see figure \ref{fig:PEG}. Furthermore PEG polymers of different sizes $1$k, $5$k, $10$k and $20$k\footnote{where k stands for kDa, the molecular weight of the polymer} are used creating particles of different total radius (GNP+PEG). The total radius is important for the diffusion of the particles through biological membranes. 

\begin{wrapfigure}{lt}{0.45\textwidth}
  \begin{center}
  \includegraphics[width=0.43\textwidth]{gfx/PEG}
\end{center}
\caption{The synthesis of methoxy-PEG$_n$ thiol}
\label{fig:PEG}
\end{wrapfigure}

Besides the targeting to the tumor, the PEG-coating also provides steric repulsion to prevent aggregation of the nanoparticles. To find out how much PEG is needed to form a stable colloid, a saturated NaCl solution is added to the functionalised GNP. The NaCl causes aggregation of the nanoparticles which can be prevented if there is enough PEG around the particle. Measurements of the size before and after NaCl addition for different proportions of PEG vs. GNP are compared. If little difference is observed between the two, there is enough PEG to provide the necessary steric repulsion, forming a stable colloid.

\subsection{Characterization}
\subsubsection{Core Diameter}
\label{subsubsec:CoreDia}
\begin{wrapfigure}{r!}{0.35\textwidth}
  \begin{center}
  \includegraphics[width=0.33\textwidth]{gfx/TEM2.png}
\end{center}
\caption{A schematic representation of a transmission electron microscope.}
\label{fig:TEM}
\end{wrapfigure}
To measure the core diameter of the gold nanoparticle, images of the colloid are created with a transmission electron microscope (TEM). With this technique it is also possible to get an idea of the shape of the nanoparticles and the polydispersity of the sample.\\
The resolving power of a microscope is limited by the wavelength of the used radiation. For optical microscopes the resolving power is approximately $0.2\mu$m \cite{celbio}. By using electrons instead of photons and magnetic lenses instead of optical lenses, much better resolutions can be achieved. That is because the de Broglie wavelength of electrons can be much smaller than the wavelength of optical photons. The resolving power of a TEM is about $0.1$nm. This is larger than the theoretical calculated resolving power and is mainly due to the limited performance of the magnetic lenses \cite{FEI}.\\
In a TEM the electron beam is produced by heating the cathode, which is a tungsten filament. By applying a voltage difference between the cathode and anode, the electrons are accelerated. They can leave the electron gun trough a hole in the anode. The electron beam is then focused on the object by the condenser lens. After passing trough the specimen, the transmitted electrons are focused onto a viewing device by the objective lens. The electron image must be made visible to the eye. This can be done with a fluorescent screen or the image can be captured digitally with a CCD camera for display on a computer \cite{TEM}.\\
The TEM image is constructed by the variation in electron transmission. Only the gold core is visible with an electron microscope, because for the PEG-layer the cross-section for scattering of absorption of the electrons is too low. therefore no difference can be observed between functionalised an non-functionalised particles using a transmission electron microscop \cite{Goldbul}.\\
For the measurements a CM-200 FEG Philips transmission electron microscope is used. The images are analyzed with the image processing program ImageJ to obtain the size distribution of the particles. From this the average diameter and a statistical error are calculated.

\begin{figure}
\centering
 \includegraphics[width=0.45\textwidth]{gfx/plasmon}
 \caption{Schematic representation of surface plasmons}
 \label{fig:plasmon}
\end{figure}

\subsubsection{Relative size}
\label{subsubsec:RelSize}
\begin{wrapfigure}{r}{0.4\textwidth}
  \begin{center}
    \includegraphics[width=0.35\textwidth]{gfx/hydroR}
  \end{center}
  \caption{A gold nanoparticle with PEG coating to indicate the hydrodynamic radius $R_h$}
  \label{fig:hydroR}
\end{wrapfigure}

A typical characterization technique of GNP is a measurement of the abosorbance spectrum in the ultra-violet and visible spectrum; UV-Vis spectroscopy. The peak in the spectrum provides information about the core diameter of the particles. \\
When electromagnetic radiation falls in on the surface of a particle the oscillating electric field causes the surface electron cloud to oscillate in the opposite direction (see figure \ref{fig:plasmon}). One side of the particle becomes negatively charged whereupon the other side becomes positive since the particles are netto neutral. This distribution of charge establishes a restoring force which causes the electron cloud to oscillate back, this time with a frequency depending on the geometrical properties of the particle; the natural surface plasmon frequency. If this frequency is equal to the frequency of the incident radiation the resonance condition is satisfied creating a peak in the absorbance spectrum. \\
From the quantitative result of the absorbance peak approximate, results for the core diameter of the particles can be obtained. In this project the absorbance spectra of different particles will be compared to analyze the size of the samples relative to each other. A smaller particle will create a higher restoring force which results in a higher surface plasmon frequency (see figure )%#TODO.
To measure the absorption spectra the %#TODO apparaat
is used and the wavelength of the incident radiation is variated between $300$nm and $900$nm in steps of $1$nm. The measured absorption for each wavelength is an average of $25$ flashes.
This technique will also be used to determine the necessary proportion PEG vs. GNP to create a stable colloid, see section \ref{subsubsec:Funct}. If after the addition of NaCl the absorbance peak has shifted to higher wavelengths (lower frequencies) or a secondary peak arises, aggregation of the nanoparticles has taken place and more PEG is needed. The colloids with the smallest particles have the most surface to fill thus need the most PEG to be stabilized. therefore, these experiments will be done for each of the four different PEG's only for the smallest particles. Once the optimal proportion (for each PEG) is determined for the smallest particles, this proportion will also work for the larger ones.
\\In this project a $1$g/l PEG solution is used and the proportion PEG/GNP is varied from $1$/$10$ to $10$/$10$ (volume to volume ratio). A well of an UV-Vis plate has a maximum volume of $200\mu$l. For all measurements with NaCl, $50\mu$l of a saturated NaCl solution is used. For the lowest proportions ($1/10$-$5/10$) $100\mu$l GNP solution is mixed with $10\mu$l to $50\mu$l PEG solution (according to the proportions). In order not to exceed the maximum volume of the UV-Vis plate, for the higher proportions ($6/10$ - $10/10$ ) only $50\mu$l GNP solution and $30\mu$l- $50\mu$l PEG solution is added. This will lower the height of the absorption peak. But since the significant physical information is contained in the position (resonance frequency) and not the height of the peak, no problems should occur while interpreting the results.


\subsubsection{Hydrodynamic Radius}
For the practical applications the GNP will be coated with a PEG layer for targeting to the tumor. It is important to know the total radius of the particle (gold plus coating) since this will determine the diffusive properties of the particles.
The total radius of the particle plus layer (gold $+$ PEG or gold $+$ H$_2$O) is called the hydrodynamic radius (R$_h$), see figure \ref{fig:hydroR}. This radius can be measured using the dynamic light scattering (DLS) technique.
The DLS technique is based on the Rayleigh scattering of incident infrared light by the gold nanoparticles. \\ 
%TO DO: uitleggen dat er in oplossing deeltjes van het oplosmiddel gaan 'kleven' aan de GNP
The GNP in the colloid perform a brownian motion. Due to this random motion the total intensity of the scattered light will vary over time. If the particles have a smaller R$_h$ they will move faster and the total intensity will vary accordingly. The variation of the total intensity over time is thus a measure of the hydrodynamic radius of the particles. \\

\begin{figure}
  \centering
  \begin{subfigure}[t]{0.48\textwidth}
    \includegraphics[width=\textwidth]{gfx/DLSint}
    \caption{}
    \label{fig:DLSint}
  \end{subfigure}
	\quad
  \begin{subfigure}[t]{0.48\textwidth}
    \includegraphics[width=\textwidth]{gfx/DLScor}
    \caption{}
    \label{fig:DLScor}
  \end{subfigure}
  \caption{The total scattered intensity (a) and autocorrelation function (b) for two different particles. Due to faster brownian motion of the smaller particle it's intensity variation in time is higher and consequently have a steeper autocorrelation function.}
\end{figure}
\begin{equation}
  \mean{R^2(t)}=6Dt
\end{equation}
with D the diffusion constant
\begin{equation}
  D=\frac{k_bT}{6\pi\eta R_h}
\end{equation}
This random motion causes the total scattered intensity $I(t)$ to fluctuate over time and a normalized autocorrelation function is defined which compares the intensity at time t with the intensity a time interval $\tau$ later (see figure).
%#TODO
\begin{equation}
  g_2(\tau)=\frac{\mean{I(t)I(t+\tau)}}{\mean{I(t)}^2}
  \label{eq:autocor}
\end{equation}
For a monodisperse sample this function can be written in function of the diffusion constant
%http://www.wyatt.com/library/theory/dynamic-light-scattering-theory.html
\begin{equation}
  g_2(\tau)=1+\beta|g_1(q,\tau)|^2 \quad \text{with }g_1(q,\tau)=\text{exp}(-q^2\mean{R^2(\tau)})=\text{exp}(-q^26D\tau)
  \label{eq:autocorR}
\end{equation}
where $\beta$ is an instrumental factor and $q$ the wavevector of the scattered light \cite{DLSBook} % #TODO
\begin{equation}
  q=\frac{4 \pi n}{\lambda_0}\text{sin}(\frac{\theta}{2})
\end{equation}
with $n$ the refractive index of the medium, $\lambda_0$ the wavelength of the incident radiation and $\theta$ the scattering angle.
Equation \eqref{eq:autocorR} clearly shows that if the particles are smaller the autocorrelation function will be steeper (see figure \ref{fig:DLScor}). This is in agreement with \textbf{corresponds to} the more rapid variation of the scattered intensity for smaller particles.
For a polydisperse sample the same equation holds only now $g_1(q,\tau)$ is the sum of all exponential decays measured in the autocorrelation function.
The specific analysis is executed by the \emph{Vasoc- particle size analyzer} Nano-Q software using the Pade-Laplace method \cite{DLSManual}.%#TODO 
One ml of GNP colloid is analyzed 5 times. Based on the quality of the fitting of the autocorrelation function, measurements are accepted or rejected. This is repeated \textbf{done twice} for each sample so in total there are $10$ measurements for each sample (if none are rejected).


\subsubsection{Stability of the colloid}
There are several ways to create a stable colloid, to prevent the particles from aggregating. One way is to create steric repulsion by coating the particles with large molecules (mostly polymers). As mentioned in the previous sections the GNP in this project will be coated with a methoxy-PEG$_n$ thiol layer for optimal targeting to the tumor. This PEG layer immediately aids the stability of the colloid.  Another way to stabilize the colloid is by using electrostatic repulsion. If all particles have an equal and high surface potential the coulomb repulsion will stop them from aggregating.
Measurements of the surface potential is a typical characterization of gold nanoparticles and will be performed in this project.
When no functionalization has happened, the citrate ions (see section \ref{subsubsec:Funct}) surround the particles and the surface potential should be negative. If the particles are coated with an methoxy-PEG$_n$ thiol layer no surface potential should be present.\\
As figure \ref{fig:zeta} shows the naked gold particles in solution have one tightly bound layer of negative ions (citrate ions) and a second less tightly bound layer of positive and negative ions. The surface potential of this second layer will determine the electrostatic properties of the particles, it is called the $\zeta$-potential. In general it is said that if the absolute value of the $\zeta$-potential is larger than $30mV$ the colloid is stable. That is, there is enough electrostatic repulsion to prevent the particles from aggregating. \\
\begin{wrapfigure}{r}{0.45\textwidth}
  \begin{center}
  \includegraphics[width=0.4\textwidth]{gfx/zeta}
  \end{center}
  \caption{Schematic representation of the double layer configuration of particles in a colloid.}
  \label{fig:zeta}
\end{wrapfigure}

To measure the $\zeta$-potential the Laser-Dopler electrophoresis technique is used. When an electric field is applied over a sample the charged particles start to accelerate. Since the particles are in solution they undergo a drag force and will eventually move at a constant velocity $v$.
\begin{equation}
  E\cdot q=\alpha \cdot v \Rightarrow v=\mu_e\cdot E \quad \text{with }\mu_e=\frac{q}{\alpha}
\end{equation}
Here $\mu_e$ is called the electrophoretic mobility and its link with the $\zeta$-potential in the Smoluchowki approximation is given by the following equation.
\begin{equation}
  \zeta=\frac{2 \eta \mu_e}{3\epsilon}=\frac{2 \eta v}{3\epsilon E}
\end{equation}
with $\eta$ the viscosity and $\epsilon$ the dielectric constant of the medium. Measuring the constant velocity of the particles in electric field E is done by irradiating the moving particles with a laser of known wavelength and registrate the doppler shift in the scattered radiation. Hence an experimental value for the $\zeta$-potential is obtained and this is a quantitative indicator for the stability of the colloid.
\subsection{Radiosensitization}
\subsubsection{Sample Preparation}
To verify the effect of Gold nanoparticles on DNA a mixture of DNA and GNP solution (PEGylated) is prepared. These samples will then be irradiated with varying X-Ray dose ($0$Gy-$15$Gy). To have an optimal radiosensitization effect on the DNA irradiation the GNP solutions have to be centrifuged. The concentration of gold nanoparticles in a $250$ ml solution obtained following the protocol described in the appendix is $1.7\cdot10^9$ GNP/$\mu$l, $2.1 \cdot10^8$ GNP/$\mu$l and $6.3 \cdot10^7$ GNP/$\mu$l for the particles of size $15$nm, $30$nm and $45$nm respectively. This has been calculated using the known molar masses of gold and tetrachlorauric acid, the mass density of gold ($\rho_{Au}=19.32$) and under the assumption that all particles have the same size and are spherical.
The total volume is divided in four equal parts and each part is functionalized with a different PEG. After centrifugation each volume is reduced to $0.5$ml achieving a concentration for the $15$nm, $30$nm and $45$nm particles of $2.1\cdot10^{11}$ GNP/$\mu$l, $2.7\cdot10^{10}$ GNP/$\mu$l and $7.9\cdot10^{9}$ GNP/$\mu$l respectively.
The DNA being used is a supercoiled double-stranded plasmid of $5386$ basepair long and a molecular weight of $3.50\cdot10^6$Da \footnote{phiX$174$ from Thermo Scientific}. $2.5\mu$l of $0.5\mu$g/$\mu$l DNA solution is mixed with $30\mu$l of the centrifuged GNP solutions. To prevent the DNA from deterioration $17.5\mu$l of phosphate buffered saline (PBS)\footnote{from Sigma-ALdrich} was added to each sample.  \\
The final samples are a $50\mu$l solution with roughly a $30:1$ GNP:DNA ratio for the $15$nm particles, a $4:1$ ration for the $30$nm particles and a $1:1$ ratio for the $45$nm particles.
\subsubsection{DNA damage analysis}
The irradiation of the samples was done using a Baltograph XSD $225$ X-ray generator from Balteau NTD at 199kVp. The $12$ samples are treated with different doses: $0$Gy, $5$Gy, $10$Gy and $15$Gy. \\
After the irradiation, the damage on the DNA due to the X-ray radiation is analyzed. As discussed in section \ref{subsec:bioeffects} the effect of ionizing radiation on DNA can be a single strand break or a double strand break, but double strand breaks are the most favorable for damaging tumor cells, because their repair mechanisms are the most complex. When a single strand break occurs in supercoiled DNA it becomes circular. A double strand break has linear DNA as a result. \\
The analysis of the DNA damage is done using gel electrophoresis. With this technique it is possible to separate charged macromolecules based on their charge and size. The charged molecules that need to be separated are put into a well at one end of a gel. Then a voltage is applied across the gel and the charged molecules move trough the gel. Smaller molecules and molecules with a larger charge will move faster trough the gel, whereas larger molecules and molecules with a lower charge will move slower. Molecules with the same charge and size will travel the same distance trough the gel and will collect into a band.

\begin{wrapfigure}{l}{0.5\textwidth}
  \begin{center}
    \includegraphics[width=0.48\textwidth]{gfx/afbgel}
  \end{center}
  \caption{An example of a gel electrophoresis result. The DNA bands correspond to circular, linear and supercoiled DNA \cite{afbgel}\cite{supercoiled}.}
	 \vspace{-10pt}
	\label{fig:afbgel}
\end{wrapfigure}

Because DNA is negatively charged, it is suitable for gel electrophoresis. Supercoiled DNA is the most compact and so has the highest mobility, therefore supercoiled DNA travels the largest distance in the gel. Circular DNA, caused by a single strand break, has the lowest mobility and therefore travels the smallest distance trough the gel, due to its large expanse. Linear DNA which is a result of a double strand break travels an intermediate distance. Figure \ref{fig:afbgel} shows an example of a gel electrophoresis image with circular, linear and supercoiled DNA. \\  
Here a voltage of $120$V is applied across an $1\%$ agarose gel during $35$ minutes. Afterward the gel is colored with special dye that binds to the DNA and is active when exposed to UV-light. Then imaging of the gel electrophoresis experiment is done using the ... transilluminator. The amount of DNA damage is analyzed by determining the intensity of the different bands with the use of the program ImageJ. 
