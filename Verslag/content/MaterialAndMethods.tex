\section{Materials and Methods}
The project consists of three major parts. First the gold nanoparticles are synthesized using the Turkevich \cite{Turkevich} method followed by thorough characterization of the particles using several techniques such as TEM, UV-VIS, DLS and $\zeta$-potential. Finally the radiosensitization effect of the GNP is tested by irradiation of samples containing circular DNA and GNP. \\
This section gives an overview of all techniques and their mechanism used in this project.
\subsection{Synthesis GNP}
\subsubsection{Method of Turkevich}
\label{subsubsec:Turk}
The synthesis of gold nanoparticles is a two step process. First there is the nucleation step. Small seeds of atomic gold are formed after the addition of an reducing agent to a solution of gold ions in an aqueous environment.  Secondly there's the growth process where small particles aggregate together to form bigger particles. \\
\begin{wrapfigure}{r!}{0.6\textwidth}
  \begin{center}
  \includegraphics[width=0.55\textwidth]{gfx/reduction}
\end{center}
\caption{The synthesis of GNP using the sodiumcitrate (NA$_3$C$_6$H$_5$O$_7$) as an reducing agent.}
\label{fig:reduction}
\end{wrapfigure}
As mentioned above the Turkevich method is used. Sodiumcitrate (NA$_3$C$_6$H$_5$O$_7$) is added to an tetrachlorauric acid (HAuCl$_4$) aqueous solution to reduce the Au$^{3+}$ ions (see figure \ref{fig:reduction}),the nucleation process has started. After the reduction the negative citrate ions remain on the surface of the newly synthesized gold seed causing a negative surface potential. At first the electrostatic repulsion is low and the Van der Waals forces cause the seeds to aggregate and form bigger particles. During this growth process more and more citrate ions cover the surface of the particles and eventually establish an sufficiently high electrostatic potential to prevent further growth of the nanoparticles. That way the sodiumcitrate both starts and ends the synthesis of the nanoparticles, controlling the size of the nanoparticles. Adding more citrate to the tetrachlorauric acid solution will stop the growth process sooner creating smaller nanoparticles.
For this project and the further medical applications it's interesting to do the experiments (functionalization, characterization and radiosensitization) for particles of different sizes. Following the method described in the appendix particles of size $15$nm, $30$nm and $45$nm are synthesized.
\subsubsection{Functionalisation with PEG}
\label{subsubsec:Funct}
\begin{wrapfigure}{l!}{0.5\textwidth}
  \begin{center}
  \includegraphics[width=0.43\textwidth]{gfx/PEG}
\end{center}
\caption{The synthesis of methoxy-PEG$_n$ thiol}
\label{fig:PEG}
\end{wrapfigure}
In order to have an optimal targeting of the GNP to the tumor the particles are coated with an layer of PEG derivates (Polyethyleen glycol). One of the end hydroxyl groups is substituted by an sulfhydryl group (SH). This substitution ensures an favorable PEG-GNP bound.
In an aqueous environment this group deprotonates forming an radical (RS$\cdot$). The PEG$_n$ thiol is now negatively charged. It will substitute the citrate ions on the surface of the GNP since an thiolate–gold bound is comparable in strength to that of the gold-gold bound\cite{gold-surfur}. The second hydroxyl substitution will determine the surface properties of the particles in the colloid. In this project we substitute with an methyl-group creating an neutral surface potential.
Thus, the GNP are coated with an layer of Methoxy-PEG$_n$ thiol, see figure \ref{fig:PEG}. Furthermore PEG polymers of different sizes $1$k, $5$k, $10$k and $20$k\footnote{where k stands for kDa, the molecular weight of the polymer} are used creating particles of different total radius (GNP+PEG). The total radius is important for the diffusion of the particles through biological membranes. \\
Besides the targeting to the tumor the PEG-coating also provides steric repulsion to prevent aggregation of the nanoparticles. To find out how much PEG is needed to form a stable colloid an saturated NaCl solution is added to the functionalised GNP. The NaCl causes aggregation of the nanoparticles which can be prevented if there's enough PEG around the particle. Measurements of the size before and after NaCl addition for different proportions of PEG vs. GNP are compared. If (almost) no difference is observed between the two there's enough PEG to provide the necessary steric repulsion forming a stable colloid.

\subsection{Characterization}
\subsubsection{Core Diameter}
\label{subsubsec:CoreDia}
\begin{wrapfigure}{r!}{0.5\textwidth}
  \begin{center}
  \includegraphics[width=0.45\textwidth]{gfx/TEM.png}
\end{center}
\caption{A schematic representation of a transmission electron microscope.\cite{TEM}}
\label{fig:TEM}
\end{wrapfigure}
To measure the core diameter of the gold nanoparticle, images of the colloid are created with a transmission electron microscope (TEM). With this technique it is also possible to get an idea of the shape of the nanoparticles and the polydispersity of the sample.\\
The resolving power of a microscope is limited by the wavelength of the used radiation. For optical microscopes the resolving power is approximately $0.2\mu$m.\cite{celbio} By using electrons instead of photons and magnetic lenses instead of optical lenses, much better resolutions can be achieved. That is because the de Broglie wavelength of electrons can be much smaller than the wavelength of optical photons. The resolving power of a TEM is about $0.1$nm. This is larger than the theoretical calculated resolving power and is mainly due to the limited performance of the magnetic lenses. \cite{FEI}\\
In a TEM the electron beam is produced by heating the cathode, which is a tungsten filament. By applying a voltage difference between the cathode and anode, the electrons are accelerated. They can leave the electron gun trough a hole in the anode. The electron beam is then focused on the object by the condenser lens. After passing trough the specimen, the transmitted electrons are focused onto a viewing device by the objective lens. The electron image must be made visible to the eye. This can be done with a fluorescent screen or the image can be captured digitally for display on a computer.\cite{TEM}\\
The TEM image is constructed by the variation in electron transmission. Only the gold core is visible with an electron microscope, because for the PEG-layer the cross-section for scattering of absorption of the electrons is too low. Therefore no difference can be observed between functionalised an non-functionalised particles using a transmission electron microscope.\cite{Goldbul}\\
For the measurement a ... transmission electron microscope is used. The images are analyzed with the image processing program ImageJ to obtain the size distribution of the particles. From this the average diameter and a statistical error are calculated.
\begin{figure}{r}
\centering
 \includegraphics[width=0.45\textwidth]{gfx/plasmon}
 \caption{Schematic representation of surface plasmons}
 \label{fig:plasmon}
\end{figure}

\subsubsection{Relative size}
A typical characterization technique of GNP is a measurement of the abosorbance spectrum in the ultra-violet and visible spectrum; UV-VIS spectroscopy. The peak in the spectrum provides information about the core diameter of the particles. \\

When electromagnetic radiation falls in on the surface of a particle the oscillating electric field causes the surface electron cloud to oscillate in the opposite direction (see figure \ref{fig:plasmon}). One side of the particle becomes negatively charged whereupon the other side becomes positive since the particles are netto neutral. This distribution of charge establishes a restoring force which causes the electron cloud to oscillate back, this time with a frequency depending on the geometrical properties of the particle; the natural surface plasmon frequency. If this frequency is equal to the frequency of the incident radiation the resonance condition is satisfied creating a peak in the absorbance spectrum. \\
From the quantitative result of the absorbance peak approximate results for the diameter of the particles can be obtained. These are in relatively good agreement with results from TEM images \cite{uv-vis-size}. In this project the absorbance spectra of different particles will be compared to analyze the size of the samples relative to each other. A smaller particle will create a higher restoring force which results in a higher surface plasmon frequency (see figure )%#TODO.
\begin{wrapfigure}{r}{0.4\textwidth}
  \begin{center}
    \includegraphics[width=0.35\textwidth]{gfx/hydroR}
  \end{center}
  \caption{A gold nanoparticle with PEG coating to indicate the hydrodynamic radius $R_h$}
  \label{fig:hydroR}
\end{wrapfigure}
This technique will also be used to determine the necessary proportion PEG vs. GNP to create a stable colloid, see section \ref{subsubsec:Funct}. If after the addition of NaCl the absorbance peak has shifted to higher wavelengths (lower frequencies) or a secondary peak arises, aggregation of the nanoparticles has taken place and more PEG is needed.

\subsubsection{Hydrodynamic Radius}
For the practical applications the GNP will be coated with a PEG layer for targetting to the tumor. It's important to know the total radius of the particle (Gold plus coating) since this will determine the diffusive properties of the particles.
The total radius of the particle plus layer (gold $+$ PEG or gold $+$ H$_2$O) is called the hydrodynamic radius (R$_h$), see figure \ref{fig:hydroR}. This radius can be measured using the dynamic light scattering (DLS) technique
The DLS technique is based on the Rayleigh scattering of incident infrared light by the gold nanoparticles. \\ The GNP in the colloid perform a brownian motion. Due to this random motion the total intensity of the scattered light will vary over time. If the particles have a smaller R$_h$ they will move faster and the total intensity will vary accordingly. The variation of the total intensity over time is thus a measure of the hydrodynamic radius of the particles. \\
\begin{figure}
  \centering
  \begin{subfigure}[t]{0.48\textwidth}
    \includegraphics[width=\textwidth]{gfx/DLSint}
    \caption{}
    \label{fig:DLSint}
  \end{subfigure}
	\quad
  \begin{subfigure}[t]{0.48\textwidth}
    \includegraphics[width=\textwidth]{gfx/DLScor}
    \caption{}
    \label{fig:DLScor}
  \end{subfigure}
  \caption{The total scattered intensity (a) and autocorrelation function (b) for two different particles. Due to faster brownian motion of the smaller particle it's intensity variation in time is higher and consequently have a steeper autocorrelation function.}
\end{figure}
\begin{equation}
  \mean{R^2(t)}=6Dt
\end{equation}
with D the diffusion constant
\begin{equation}
  D=\frac{k_bT}{6\pi\eta R_h}
\end{equation}
This random motion causes the total scattered intensity $I(t)$ to fluctuate over time and a normalized autocorrelation function is defined which compares the intensity at time t with the intensity a time interval $\tau$ later (see figure).
%#TODO
\begin{equation}
  g_2(\tau)=\frac{\mean{I(t)I(t+\tau)}}{\mean{I(t)}^2}
  \label{eq:autocor}
\end{equation}
For a monodisperse sample this function can be written in function of the diffusion constant
%http://www.wyatt.com/library/theory/dynamic-light-scattering-theory.html
\begin{equation}
  g_2(\tau)=1+\beta|g_1(q,\tau)|^2 \quad \text{with }g_1(q,\tau)=\text{exp}(-q^2\mean{R^2(\tau)})=\text{exp}(-q^26D\tau)
  \label{eq:autocorR}
\end{equation}
where $\beta$ is an instrumental factor and $q$ the wavevector of the scattered light \cite{DLSBook} % #TODO
\begin{equation}
  q=\frac{4 \pi n}{\lambda_0}\text{sin}(\frac{\theta}{2})
\end{equation}
with $n$ the refractive index of the medium, $\lambda_0$ the wavelength of the incident radiation and $\theta$ the scattering angle.
Equation \ref{eq:autocorR} clearly shows that if the particles are smaller the autocorrelation function will be steeper (see figure \ref{fig:DLScor}). This is in agreement with the more rapid variation of the scattered intensity for smaller particles.
For a polydisperse sample the same equation holds only now $g_1(q,\tau)$ is the sum of all exponential decays measured in the autocorrelation function.
The specific analysis is executed by the \emph{Vasoc- particle size analyzer} Nano-Q software using the Pade-Laplace method \cite{DLSManual}.


\subsubsection{Stability of the colloid}
There are several ways to create a stable colloid, to prevent the particles from aggregating. One way is to create steric repulsion by coating the particles with big molecules (mostly polymers). As mentioned in the previous sections the GNP in this project will be coated with a methoxy-PEG$_n$ thiol layer for optimal targeting to the tumor. This PEG layer immediately aids the stability of the colloid.  Another way to stabilize the colloid is by using electrostatic repulsion. If all particles have an equal and high surface potential the coulomb repulsion will stop them from aggregating.
Measurements of the surface potential is a typical characterization of gold nanoparticles and will be performed in this project.
When no functionalization has happened the citrate ions (see section \ref{subsubsec:Funct}) surround the particles and the surface potential should be negative. If the particles are coated with an methoxy-PEG$_n$ thiol layer no surface potential should be present.\\
\begin{wrapfigure}{r}{0.45\textwidth}
  \begin{center}
  \includegraphics[width=0.4\textwidth]{gfx/zeta}
  \end{center}
  \caption{Schematic representation of the double layer configuration of particles in a colloid.}
  \label{fig:zeta}
\end{wrapfigure}

As figure \ref{fig:zeta} shows the naked gold particles in solution have one tightly bound layer of negative ions (citrate ions) and a second less tightly bound layer of positive and negative ions. The surface potential of this second layer will determine the electrostatic properties of the particles, it is called the $\zeta$-potential. In general it is said that if the $\zeta$-potential is bigger than $30mV$ the colloid is stable. That is, there's enough electrostatic repulsion to prevent the particles from aggregating. \\
To measure the $\zeta$-potential the Laser-Dopler electrophoresis technique is used. When an electric field is applied over a sample the charged particles start to accelerate. Since the particles are in solution they undergo a drag force and will eventually move at a constant velocity $v$.
\begin{equation}
  E\cdot q=\alpha \cdot v \Rightarrow v=\mu_e\cdot E \quad \text{with }\mu_e=\frac{q}{\alpha}
\end{equation}
Here $\mu_e$ is called the electrophoretic mobility and it's link with the $\zeta$-potential in the Smoluchowki approximation is given by the following equation.
\begin{equation}
  \zeta=\frac{2 \eta \mu_e}{3\epsilon}=\frac{2 \eta v}{3\epsilon E}
\end{equation}
with $\eta$ the viscosity and $\epsilon$ the dielectric constant of the medium. Measuring the constant velocity of the particles in electric field E is done by irradiating the moving particles with a laser of known wavelength and registrate the doppler shift in the scattered radiation. Hence an experimental value for the $\zeta$-potential is obtained and this is a quantitative indicator for the stability of the colloid.
\subsection{Radiosensitization}
\subsubsection{}
