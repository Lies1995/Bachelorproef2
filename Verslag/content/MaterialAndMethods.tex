\section{Theoretical backround}
The project consists of three major parts. First the gold nanoparticles are synthesized using the Turkevich \cite{Turkevich} method followed by thorough characterization of the particles using several techniques such as TEM, UV-VIS, DLS and $\zeta$-potential. Finally the radiosensitization effect of the GNP is tested by irradiation of samples containing circular DNA and GNP. \\
This section gives an overview of all techniques and their mechanism used in this project.
\subsection{Synthesis GNP}
\subsubsection{Method of Turkevich}
The synthesis of gold nanoparticles is a two step process. First there is the nucleation step. Small seeds of atomic gold are formed after the addition of an reducing agent to a solution of gold ions in an aqueous environment.  Secondly there's the growth process where small particles aggregate together to form bigger particles. \\
\begin{wrapfigure}{r!}{0.6\textwidth}
  \begin{center}
  \includegraphics[width=0.55\textwidth]{gfx/reduction}
\end{center}
\caption{The synthesis of GNP using the sodiumcitrate (NA$_3$C$_6$H$_5$O$_7$) as an reducing agent.}
\label{fig:reduction}
\end{wrapfigure}
As mentioned above the Turkevich method is used. Sodiumcitrate ($NA_3C_6H_5O_7$) is added to an tetrachlorauric acid ($H_AUCl_4$) aqueous solution to reduce the $AU^{3+}$ ions (see figure \ref{fig:reduction}). The fist seeds are formed and the nucleation process has started. After the reduction the negative citrate ions remain on the surface of the newly synthesized gold seed causing a negative surface potential. At first the electrostatic repulsion is low and the Van der Waals forces cause the seeds to aggregate and form bigger particles. During this growth process more and more citrate ions cover the surface of the particles and eventually establish an sufficiently high electrostatic potential to prevent further growth of the nanoparticles. In that way the sodiumcitrate both starts and end the synthesis of the nanoparticles, controlling the size of the nanoparticles. Adding more citrate to the tetrachlorauric acid solution will stop the growth process sooner creating smaller nanoparticles.

\subsubsection{Functionalisation with PEG}
\label{subsubsec:Funct}
\begin{wrapfigure}{l!}{0.6\textwidth}
  \begin{center}
  \includegraphics[width=0.55\textwidth]{gfx/PEG}
\end{center}
\caption{The synthesis of methoxy-PEG$_n$ thiol}
\label{fig:PEG}
\end{wrapfigure}
In order to have an optimal targeting of the GNP to the tumor the particles are coated with an layer of PEG derivates (Polyethyleen glycol). One of the end hydroxyl groups is substituted by an sulfhydryl group (SH). This substitution ensures an favorable PEG-GNP bound.
 In an aqueous environment this group deprotonates forming an radical (RS$\cdot$). The PEG$_n$ thiol is now negatively charged. It will substitute the citrate ions on the surface of the GNP since an thiolate–gold bound is comparable in strength to that of the gold-gold bound\cite{gold-surfur}. The second hydroxyl substitution will determine the surface properties of the particles in the colloid. In this project we substitute with an methyl-group creating an neutral surface potential.
Hence, the GNP are coated with an layer of Methoxy-PEG$_n$ thiol, see figure \ref{fig:PEG}.\\
Besides the targeting to the tumor the PEG-coating also provides steric repulsion to prevent aggregation of the nanoparticles. To find out how much PEG is needed to form a stable colloid an saturated NaCl solution is added to the functionalised GNP. The NaCl causes aggregation of the nanoparticles which can be prevented if there's enough PEG around the particle. Measurements of the size before and after NaCl addition for different proportions of PEG vs. GNP are compared. If (almost) no difference is observed between the two there's enough PEG to provide the necessary steric repulsion forming a stable colloid.

\subsection{Characterization}
\subsubsection{Core Diameter}
To measure the radius of the gold nanoparticle, images of the colloid are created with an transmission electron microscope (TEM). These images are then analysed to obtain a statistical estimate of the particle radius..
TO BE COMPLETED
\subsubsection{Relative size}
A typical characterization technique of GNP is a measurement of the abosorbance spectrum in the ultra-violet and visible spectrum; UV-VIS spectroscopy. The peak in the spectrum provides information about the core diameter of the particles. \\
\begin{figure}
  \centering
  \includegraphics[width=0.5\textwidth]{gfx/plasmon}
  \caption{Schematic representation of surface plasmons}
  \label{fig:plasmon}
\end{figure}
When electromagnetic radiation falls in on the surface of a particle the oscillating electric field causes the surface electron cloud to oscillate in the opposite direction (see figure \ref{fig:plasmon}). One side of the particle becomes negatively charged whereupon the other side becomes positive since the particles are netto neutral. This distribution of charge establishes a restoring force which causes the electron cloud to oscillate back, this time with a frequency depending on the geometrical properties of the particle; the natural surface plasmon frequency. If this frequency is equal to the frequency of the incident radiation the resonance condition is satisfied creating a peak in the absorbance spectrum. \\
From the quantitative result of the absorbance peak approximate results for the diameter of the particles can be obtained. These are in relatively good agreement with results from TEM images \cite{uv-vis-size}. In this project the absorbance spectra of different particles will be compared to analyze the size of the samples relative to each other. A smaller particle will create a higher restoring force which results in a higher surface plasmon frequency (see figure )%#TODO.
\\
This technique will also be used to determine the necessary proportion PEG vs. GNP to create a stable colloid, see section \ref{subsubsec:Funct}. If after the addition of NaCl the absorbance peak has shifted to higher wavelengths (lower frequencies) or a secondary peak arises, aggregation of the nanoparticles has taken place and more PEG is needed.
\subsubsection{Hydrodynamic Radius}
For the practical applications the GNP will be coated with a PEG layer both for stabilization and targetting to the tumor. It's important to know the total radius of the particle (Gold plus coating) since this will determine the diffusive properties of the particles. The total radius of the particle plus layer (gold $+$ PEG or gold $+$ H$_2$O) is called the hydrodynamic radius (R$_h$). This radius can be measured using the dynamic light scattering (DLS) technique.
The DLS technique is based on the Rayleigh scattering of incident infrared light by the gold nanoparticles. The GNP in the colloid perform a brownian motion. Due to this random motion the total intensity of the scattered light will vary over time. If the particles have a smaller R$_h$ they will move faster and the total intensity will vary faster. The variation of the total intensity over time is thus a measure of the hydrodynamic radius of the particles. \\
To obtain a quantitative result of the hydrodynamic radius a mathematical analysis is performed which first defines an normalized autocorrelation function using the measured values of the total intensity $I(t)$ over time.
\begin{equation}
  g^2(q,\tau)=\frac{I(t)I(t+\tau)>}{<I(t)>^2}
\end{equation}
where $q$ is the wavevector of the scattered light. TO BE CONTINUED
