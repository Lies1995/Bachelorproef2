%%%%%%%%%%%%%%%%%%%%%%%%%%%%%%%%%%%%%%%%
% University/School Laboratory Report
% LaTeX Template
% Version 3.1 (25/3/14)
%
% This template has been downloaded from:
% http://www.LaTeXTemplates.com
%
% Original author:
% Linux and Unix Users Group at Virginia Tech Wiki
% (https://vtluug.org/wiki/Example_LaTeX_chem_lab_report)
%
% License:
% CC BY-NC-SA 3.0 (http://creativecommons.org/licentses/by-nc-sa/3.0/)
%
%%%%%%%%%%%%%%%%%%%%%%%%%%%%%%%%%%%%%%%%%

%----------------------------------------------------------------------------------------
%	PACKAGES AND DOCUMENT CONFIGURATIONS
%----------------------------------------------------------------------------------------

\documentclass[11pt]{article}

\usepackage[version=3]{mhchem} % Package for chemical equation typesetting
\usepackage{siunitx} % Provides the \SI{}{} and \si{} command for typesetting SI units
\usepackage{wrapfig}
\usepackage{natbib} % Required to change bibliography style to APA
\usepackage[labelfont=bf]{caption}
\usepackage{subcaption}
\usepackage{amsmath,amssymb}
\usepackage{graphicx}
\usepackage{hyperref, booktabs}
\usepackage{url}
\usepackage{geometry}
\geometry{
a4paper,
left=30mm,
right=30mm,
top=20mm,
}
\usepackage{array}
\newcolumntype{x}[1]{>{\centering\arraybackslash\hspace{0pt}}p{#1}}
\setlength\parindent{0pt} % Removes all indentation from paragraphs

\renewcommand{\labelenumi}{\alph{enumi}.} % Make numbering in the enumerate environment by letter rather than number (e.g. section 6)
\def\mean#1{\left< #1 \right>}
%\usepackage{times} % Uncomment to use the Times New Roman font

%----------------------------------------------------------------------------------------
%	DOCUMENT INFORMATION
%----------------------------------------------------------------------------------------

\title{Radiosensitizatioin of cancer cells \\ using gold nanoparticles \\ \vspace{0.5cm}\small{\textsc{Bachelor Thesis}}} % Title

\author{Lies \textsc{Deceuninck}\\ Hannelore \textsc{Verhoeven}} % Author name

\date{\today} % Date for the report

\begin{document}
\maketitle % Insert the title, author and date
 \begin{center}
\begin{tabular}{l r}

Assistents: & Bert De Roo \\ % Partner names
& Mattias Vervaele \\

Professor: & Chris Van Haesendonck % Instructor/supervisor
\end{tabular}
\end{center}
\tableofcontents


 %If you wish to include an abstract, uncomment the lines below
\begin{abstract}
   Elk jaar sterven meer en meer mensen van kanker, een ziekte die ontstaat door schade aan het DNA van cellen in het lichaam. $50\%$ van de mensen die behandeld worden tegen kanker ondergaan radiatie therapie. De kwadaardige cellen, die meestal uitgegroeid zijn tot een tumor, worden met hoge energiestraling bestraald in de hoop celdood te induceren. Helaas wordt hierbij ook goedaardig biologisch materiaal vernietigd en de neveneffecten/gevolgen hievan kunnen groot zijn.
   Een recente ontwikkeling om radiatie therapie zo effectief mogelijk te maken in de radiosensitizering van de kanker cellen. In dit project is nader gekeken naar de mogelijkheden om goud nanodeeltjes (GND) hiervoor te gebruiken. De GND zijn gesynthetiseerd volgens de methode van Turkevich en gekarakteriseerd met verschillende technieken; TEM, UV-VIs, DLS and $\zeta$-potential. Er zijn deeltjes van drie verschillende grootten gesynthetiseerd; $15$nm, $30$nm en $45$nm omdat dit een belangrijke parameter is voor de diffusie eigenschappan. 
\end{abstract}

%----------------------------------------------------------------------------------------
%Content
%----------------------------------------------------------------------------------------
\newpage
\section{Introduction}
Each year 8.2 million people die from cancer, that is an estimated 13$\%$ of all deaths worldwide. Moreover the number of new cases is expected to increase by 70$\%$ over the next two decades \cite{WHO}. These numbers are a clear indication of the importance of cancer research, which includes the development of new treatments and the enhancement of existing treatments.\\
Cancer is a general term for a large group of diseases that are all characterised by a rapid creation of abnormal cells that grow beyond their usual boundaries. The cells grow out of control and thereby form tumors, because the part of the DNA responsible for cell death is disturbed.\\
The three major cancer treatments that exist today are chemotherapy, surgery and radiation therapy. In chemotherapy anti-cancer drugs are used to damage the cancer cells. These drugs only affect rapid dividing cells, which is a typical characteristic of cancer cells. Cancer can also be treated by removing the tumor by surgery. A third important cancer treatment is radiation therapy, where ionizing radiation is used to damage the DNA of the cancer cells. Because in most cases combinations of these treatments are used, it is estimated that 50$\%$ of all cancer patients undergo radiation therapy as part of their treatment \cite{imaginis}.\\
The problem with ionizing radiation is, that it does not discriminate between malignant and normal tissues. This means that normal tissue is also irradiated. Therefore radiation therapy has an unintentional toxic effect on the healthy tissue surrounding the tumor. These effects range from those that cause mild discomfort to others that are life-threatening. The radiation dose has to be carefully chosen: it has to be high enough to bring damage to the cancer cells, but it cannot be too high because of the negative effect on the healthy tissue.\\
The toxic effect of radiation therapy can be reduced in different ways. For example instead of delivering the total dose of radiation at once, the dose can be fractionated in smaller doses that are delivered over a longer period of time. Dose fractionations offers the opportunity for healthy cells to recover from the previous dose before the next dose is delivered \cite{radiobiology}. Another technique to reduce the effect on healthy tissue is radiosensitization. With this technique it is possible to make the cancer cells more sensitive to ionizing radiation, so that lower radiation doses can be used. This reduces the effect on the surrounding tissue. Here radiosensitization using gold nanoparticles (GNP) is discussed. The absorption of photons is higher for elements with high mass numbers. Therefore when the GNP can be brought inside the tumor, because of the high mass number of gold (Z=79), the ionizing radiation will mainly interact with the nanoparticles. This causes local secondary radiation, which delivers its energy locally, so inside the tumor.\cite{targeting} The reason gold is used instead of other element with high atomic mass, is that gold is biocompatible, which makes it suitable for medical treatment \cite{biocompatible}. \\
The first experimental evidence of the use of GNP to enhance radiation therapy was provided by Hainfeld et al. \cite{expevidence}. Mice with cancer tumors were injected with 1.9 nm diameter GNP and then radiated with 250 kVp X-rays. The combination of GNP and radiation resulted in a one-year survival of 86$\%$ compared to 20$\%$ with radiation therapy alone. Other experiments showed similar evidence of the radiosensitizing effect of GNP \cite{expevidence2}\cite{expevidence1}. These results provide a motivation for further research within the field of nanoparticle enhanced radiation therapy.\\
The main goal of this project is to synthesize GNP of different sizes and to characterize them using different methods. Next the GNP are functionalized with a polyethylene glycol (PEG) coating, which increases the probability of delivering the nanoparticles to the cancer cells. The PEG coating also provides stability to the GNP solution, i.e., prevents them from aggregating. Finally a mixture of DNA and GNP is irradiated and the effect on the DNA is analysed.\\
%#TODO effect PEG omdraaien? we voegen het to voor de targetting en merken dat het een positief effect heeft op de stabilizatie?
\newpage
\section{Theoretical background}
\subsection{Radiation physics}
The ionizing radiation type used for radiation therapy can vary ranging from photons and electrons to protons, neutrons and low-mass ions, but photons are by far the most common form of radiation used in cancer treatment \cite{radtype}. There are three possible ways photons can interact with matter: photoelectric absorption, Compton scattering and pair production.\\

\begin{figure}
    \centering
    \begin{subfigure}[h]{0.5\textwidth}
        \includegraphics[width=\textwidth]{gfx/compton}
        \caption{Compton effect}
        \label{fig:gull}
    \end{subfigure}
    ~ %add desired spacing between images, e. g. ~, \quad, \qquad, \hfill etc. 
      %(or a blank line to force the subfigure onto a new line)
    \begin{subfigure}[h]{0.5\textwidth}
        \includegraphics[width=\textwidth]{gfx/photoel}
        \caption{Photoelectric effect}
        \label{fig:tiger}
    \end{subfigure}
    ~ %add desired spacing between images, e. g. ~, \quad, \qquad, \hfill etc. 
    %(or a blank line to force the subfigure onto a new line)
    \begin{subfigure}[h]{0.4\textwidth}
        \includegraphics[width=\textwidth]{gfx/pair_production}
        \caption{Pair production}
        \label{fig:mouse}
    \end{subfigure}
    \caption{Possible interaction types for photons.}\label{fig:inttypes}
\end{figure}
In the photoelectric effect, the energy of an incoming photon is transferred to an electron, which is then ejected. The vacancy left by this electron is then filled with another electron from a higher shell. This electron then gives of its excess energy as an characteristic X-ray photon. In some cases the excess energy may be transferred to an outer-shell electron. As a consequence this electron is ejected and is called an Auger electron. The cross section for photoelectric absorption $\tau$ increases for increasing mass number $Z$ and in decreases sharply with the photon energy $E_{\gamma}$:
\begin{equation}
\tau \cong C^{te} \cdot \frac{Z^n}{E_{\gamma}{^{3.5}}}
\end{equation}
with $n$ varying between $4$ and $5$ \cite{Knoll}.\\
The Compton effect is an inelastic scattering between a photon and an electron, where part of the energy of the incoming photon is transferred to the recoiling electron. The cross section for Compton scattering grows linearly with $Z$ and falls of gradually with increasing energy \cite{Knoll}.\\
With pair production, a photon creates an electron-positron pair. The cross section for this process varies approximately with $Z^2$ and increases for increasing energy \cite{Knoll}.\\
Since the cross sections for each of these processes increases with increasing $Z$, it is clear that gold, with its high atomic mass, is suitable for radiosensitization.
\subsection{Biological effects}
\subsection{Targeting}

In order to have a beneficial effect of GNP in radiation therapy, it is important to bring the nanoparticles as close as possible to the DNA of the cancer cells. The uptake of GNP into the nucleus of the cells is only possible below a certain upper size limit \cite{Ken}.\\  Therefore the size of the nanoparticle is a very important parameter.
GNP are known to passively accumulate in cancer cells because of the enhanced permeability and retention (EPR) effect. Because cancer cells are rapid growing cells, tumors have leaky, immature vasculature, so that their blood vessels are more permeable \cite{Jain}. \\  This effect can be enhanced by functionalizing the naked gold nanoparticles with PEG. This PEG coating sterically hinders nonspecific binding of proteins to the surface of the particle and delays the recognition of the particles by the reticuloendothelial system. This increase the circulation time of the GNP in the blood and as a result increases the probability of delivering the nanoparticles to the tumor \cite{Ken}. \\  The hydrodynamic radius of the particles increases because of this PEG coating, therefore the optimal size (of the naked GNP) for uptake is smaller when the particle is functionalized. Finally the functionalization also has a positive effect on the stability of the GNP solution.\\
Besides PEGylation, the GNP can also be coated with antibodies, which actively bind to receptors that are specific for cancer cells. An example of a receptor that is overexpressed in tumors is the epidermal growth factor receptor. Nanoparticles coated with an antibody that corresponds to this receptor are guided to the tumor and bind on its surface \cite{Tiwari}.

% Bronnen
% WHO http://www.who.int/mediacentre/factsheets/fs297/en/
% imaginis http://www.imaginis.com/radiotherapy/cancer-treatment-with-radiation-therapy
% radiobiology http://citeseerx.ist.psu.edu/viewdoc/download?doi=10.1.1.462.2846&rep=rep1&type=pdf
% biocompatible http://www.ncbi.nlm.nih.gov/pubmed/16262332
% targeting: circulation time http://www.ncbi.nlm.nih.gov/pmc/articles/PMC3473940/
% first experimental evidence (proxy) expevidence http://iopscience.iop.org/article/10.1088/0031-9155/49/18/N03/meta

\section{Materials and Methods}
The project consists of three major parts. First the gold nanoparticles are synthesized using the Turkevich method \cite{Turkevich} followed by thorough characterization of the particles using several techniques: TEM, UV-VIS, DLS and $\zeta$-potential. Finally the radiosensitization effect of the GNP is tested by irradiating samples of circular DNA with GNP. \\
This section gives an overview of all techniques and their mechanism used in this project.
\subsection{Synthesis GNP}
\subsubsection{Method of Turkevich}
\label{subsubsec:Turk}
The synthesis of gold nanoparticles is a two step process.  Small seeds of atomic gold are formed after the addition of an reducing agent to an aqueous solution of gold ions, this is the nucleation step. Secondly there is the growth process where small particles aggregate together to form larger particles.

\begin{wrapfigure}{r!}{0.6\textwidth}
  \begin{center}
  \includegraphics[width=0.55\textwidth]{gfx/reduction}
\end{center}
\caption{The synthesis of GNP using the sodiumcitrate (NA$_3$C$_6$H$_5$O$_7$) as an reducing agent.}
\label{fig:reduction}
\end{wrapfigure} 

As mentioned above the Turkevich method is used. Sodiumcitrate (NA$_3$C$_6$H$_5$O$_7$) is added to an tetrachlorauric acid (HAuCl$_4$) aqueous solution to reduce the Au$^{3+}$ ions (see figure \ref{fig:reduction}), the nucleation process has started. After the reduction, the negative citrate ions remain on the surface of the newly synthesized gold seed producing a negative surface potential. At first the electrostatic repulsion is low and the Van der Waals forces cause the seeds to aggregate and form larger particles. During this growth process more and more citrate ions cover the surface of the particles and eventually establish a sufficiently high electrostatic potential to prevent further growth of the nanoparticles. In this manner the sodiumcitrate both starts and ends the synthesis of the nanoparticles and thereby controlling the size of the nanoparticles. Adding more citrate to the tetrachlorauric acid solution will stop the growth process sooner creating smaller nanoparticles.
For this project and the further medical applications it is interesting to do the experiments (functionalization, characterization and radiosensitization) for particles of different sizes. Particles of $15$nm, $30$nm and $45$nm are synthesized following the method described in the appendix.
\subsubsection{Functionalisation with PEG}
\label{subsubsec:Funct}
In order to have an optimal targeting of the GNP to the tumor, the particles are coated with an layer of PEG (Polyethyleen glycol) derivates. One of the end hydroxyl groups is substituted by a sulfhydryl group (SH). It substitution ensures a favorable PEG-GNP bound.
In an aqueous environment this group deprotonates, forming an radical (RS$\cdot$). The PEG$_n$ thiol is now negatively charged. It will substitute the citrate ions on the surface of the GNP since a thiolate-gold bound is comparable in strength to that of the gold-gold bound \cite{gold-surfur}. The second hydroxyl substitution will determine the surface properties of the particles in the colloid. In this project substitution with a methyl-group creating an neutral surface potential is done.
Thus, the GNP are coated with an layer of Methoxy-PEG$_n$ thiol, see figure \ref{fig:PEG}. Furthermore PEG polymers of different sizes $1$k, $5$k, $10$k and $20$k\footnote{where k stands for kDa, the molecular weight of the polymer} are used creating particles of different total radius (GNP+PEG). The total radius is important for the diffusion of the particles through biological membranes. 

\begin{wrapfigure}{lt}{0.45\textwidth}
  \begin{center}
  \includegraphics[width=0.43\textwidth]{gfx/PEG}
\end{center}
\caption{The synthesis of methoxy-PEG$_n$ thiol}
\label{fig:PEG}
\end{wrapfigure}

Besides the targeting to the tumor, the PEG-coating also provides steric repulsion to prevent aggregation of the nanoparticles. To find out how much PEG is needed to form a stable colloid, a saturated NaCl solution is added to the functionalised GNP. The NaCl causes aggregation of the nanoparticles which can be prevented if there is enough PEG around the particle. Measurements of the size before and after NaCl addition for different proportions of PEG vs. GNP are compared. If little difference is observed between the two, there is enough PEG to provide the necessary steric repulsion, forming a stable colloid.

\subsection{Characterization}
\subsubsection{Core Diameter}
\label{subsubsec:CoreDia}
\begin{wrapfigure}{r!}{0.35\textwidth}
  \begin{center}
  \includegraphics[width=0.33\textwidth]{gfx/TEM2.png}
\end{center}
\caption{A schematic representation of a transmission electron microscope.}
\label{fig:TEM}
\end{wrapfigure}
To measure the core diameter of the gold nanoparticle, images of the colloid are created with a transmission electron microscope (TEM). With this technique it is also possible to get an idea of the shape of the nanoparticles and the polydispersity of the sample.\\
The resolving power of a microscope is limited by the wavelength of the used radiation. For optical microscopes the resolving power is approximately $0.2\mu$m \cite{celbio}. By using electrons instead of photons and magnetic lenses instead of optical lenses, much better resolutions can be achieved. That is because the de Broglie wavelength of electrons can be much smaller than the wavelength of optical photons. The resolving power of a TEM is about $0.1$nm. This is larger than the theoretical calculated resolving power and is mainly due to the limited performance of the magnetic lenses \cite{FEI}.\\
In a TEM the electron beam is produced by heating the cathode, which is a tungsten filament. By applying a voltage difference between the cathode and anode, the electrons are accelerated. They can leave the electron gun trough a hole in the anode. The electron beam is then focused on the object by the condenser lens. After passing trough the specimen, the transmitted electrons are focused onto a viewing device by the objective lens. The electron image must be made visible to the eye. This can be done with a fluorescent screen or the image can be captured digitally with a CCD camera for display on a computer \cite{TEM}.\\
The TEM image is constructed by the variation in electron transmission. Only the gold core is visible with an electron microscope, because for the PEG-layer the cross-section for scattering of absorption of the electrons is too low. therefore no difference can be observed between functionalised an non-functionalised particles using a transmission electron microscop \cite{Goldbul}.\\
For the measurements a CM-200 FEG Philips transmission electron microscope is used. The images are analyzed with the image processing program ImageJ to obtain the size distribution of the particles. From this the average diameter and a statistical error are calculated.

\begin{figure}
\centering
 \includegraphics[width=0.45\textwidth]{gfx/plasmon}
 \caption{Schematic representation of surface plasmons}
 \label{fig:plasmon}
\end{figure}

\subsubsection{Relative size}
\label{subsubsec:RelSize}
\begin{wrapfigure}{r}{0.4\textwidth}
  \begin{center}
    \includegraphics[width=0.35\textwidth]{gfx/hydroR}
  \end{center}
  \caption{A gold nanoparticle with PEG coating to indicate the hydrodynamic radius $R_h$}
  \label{fig:hydroR}
\end{wrapfigure}

A typical characterization technique of GNP is a measurement of the abosorbance spectrum in the ultra-violet and visible spectrum; UV-Vis spectroscopy. The peak in the spectrum provides information about the core diameter of the particles. \\
When electromagnetic radiation falls in on the surface of a particle the oscillating electric field causes the surface electron cloud to oscillate in the opposite direction (see figure \ref{fig:plasmon}). One side of the particle becomes negatively charged whereupon the other side becomes positive since the particles are netto neutral. This distribution of charge establishes a restoring force which causes the electron cloud to oscillate back, this time with a frequency depending on the geometrical properties of the particle; the natural surface plasmon frequency. If this frequency is equal to the frequency of the incident radiation the resonance condition is satisfied creating a peak in the absorbance spectrum. \\
From the quantitative result of the absorbance peak approximate, results for the core diameter of the particles can be obtained. In this project the absorbance spectra of different particles will be compared to analyze the size of the samples relative to each other. A smaller particle will create a higher restoring force which results in a higher surface plasmon frequency (see figure )%#TODO.
To measure the absorption spectra the %#TODO apparaat
is used and the wavelength of the incident radiation is variated between $300$nm and $900$nm in steps of $1$nm. The measured absorption for each wavelength is an average of $25$ flashes.
This technique will also be used to determine the necessary proportion PEG vs. GNP to create a stable colloid, see section \ref{subsubsec:Funct}. If after the addition of NaCl the absorbance peak has shifted to higher wavelengths (lower frequencies) or a secondary peak arises, aggregation of the nanoparticles has taken place and more PEG is needed. The colloids with the smallest particles have the most surface to fill thus need the most PEG to be stabilized. therefore, these experiments will be done for each of the four different PEG's only for the smallest particles. Once the optimal proportion (for each PEG) is determined for the smallest particles, this proportion will also work for the larger ones.
\\In this project a $1$g/l PEG solution is used and the proportion PEG/GNP is varied from $1$/$10$ to $10$/$10$ (volume to volume ratio). A well of an UV-Vis plate has a maximum volume of $200\mu$l. For all measurements with NaCl, $50\mu$l of a saturated NaCl solution is used. For the lowest proportions ($1/10$-$5/10$) $100\mu$l GNP solution is mixed with $10\mu$l to $50\mu$l PEG solution (according to the proportions). In order not to exceed the maximum volume of the UV-Vis plate, for the higher proportions ($6/10$ - $10/10$ ) only $50\mu$l GNP solution and $30\mu$l- $50\mu$l PEG solution is added. This will lower the height of the absorption peak. But since the significant physical information is contained in the position (resonance frequency) and not the height of the peak, no problems should occur while interpreting the results.


\subsubsection{Hydrodynamic Radius}
For the practical applications the GNP will be coated with a PEG layer for targeting to the tumor. It is important to know the total radius of the particle (gold plus coating) since this will determine the diffusive properties of the particles.
The total radius of the particle plus layer (gold $+$ PEG or gold $+$ H$_2$O) is called the hydrodynamic radius (R$_h$), see figure \ref{fig:hydroR}. This radius can be measured using the dynamic light scattering (DLS) technique.
The DLS technique is based on the Rayleigh scattering of incident infrared light by the gold nanoparticles. \\ 
%TO DO: uitleggen dat er in oplossing deeltjes van het oplosmiddel gaan 'kleven' aan de GNP
The GNP in the colloid perform a brownian motion. Due to this random motion the total intensity of the scattered light will vary over time. If the particles have a smaller R$_h$ they will move faster and the total intensity will vary accordingly. The variation of the total intensity over time is thus a measure of the hydrodynamic radius of the particles. \\

\begin{figure}
  \centering
  \begin{subfigure}[t]{0.48\textwidth}
    \includegraphics[width=\textwidth]{gfx/DLSint}
    \caption{}
    \label{fig:DLSint}
  \end{subfigure}
	\quad
  \begin{subfigure}[t]{0.48\textwidth}
    \includegraphics[width=\textwidth]{gfx/DLScor}
    \caption{}
    \label{fig:DLScor}
  \end{subfigure}
  \caption{The total scattered intensity (a) and autocorrelation function (b) for two different particles. Due to faster brownian motion of the smaller particle it's intensity variation in time is higher and consequently have a steeper autocorrelation function.}
\end{figure}
\begin{equation}
  \mean{R^2(t)}=6Dt
\end{equation}
with D the diffusion constant
\begin{equation}
  D=\frac{k_bT}{6\pi\eta R_h}
\end{equation}
This random motion causes the total scattered intensity $I(t)$ to fluctuate over time and a normalized autocorrelation function is defined which compares the intensity at time t with the intensity a time interval $\tau$ later (see figure).
%#TODO
\begin{equation}
  g_2(\tau)=\frac{\mean{I(t)I(t+\tau)}}{\mean{I(t)}^2}
  \label{eq:autocor}
\end{equation}
For a monodisperse sample this function can be written in function of the diffusion constant
%http://www.wyatt.com/library/theory/dynamic-light-scattering-theory.html
\begin{equation}
  g_2(\tau)=1+\beta|g_1(q,\tau)|^2 \quad \text{with }g_1(q,\tau)=\text{exp}(-q^2\mean{R^2(\tau)})=\text{exp}(-q^26D\tau)
  \label{eq:autocorR}
\end{equation}
where $\beta$ is an instrumental factor and $q$ the wavevector of the scattered light \cite{DLSBook} % #TODO
\begin{equation}
  q=\frac{4 \pi n}{\lambda_0}\text{sin}(\frac{\theta}{2})
\end{equation}
with $n$ the refractive index of the medium, $\lambda_0$ the wavelength of the incident radiation and $\theta$ the scattering angle.
Equation \eqref{eq:autocorR} clearly shows that if the particles are smaller the autocorrelation function will be steeper (see figure \ref{fig:DLScor}). This is in agreement with \textbf{corresponds to} the more rapid variation of the scattered intensity for smaller particles.
For a polydisperse sample the same equation holds only now $g_1(q,\tau)$ is the sum of all exponential decays measured in the autocorrelation function.
The specific analysis is executed by the \emph{Vasoc- particle size analyzer} Nano-Q software using the Pade-Laplace method \cite{DLSManual}.%#TODO 
One ml of GNP colloid is analyzed 5 times. Based on the quality of the fitting of the autocorrelation function, measurements are accepted or rejected. This is repeated \textbf{done twice} for each sample so in total there are $10$ measurements for each sample (if none are rejected).


\subsubsection{Stability of the colloid}
There are several ways to create a stable colloid, to prevent the particles from aggregating. One way is to create steric repulsion by coating the particles with large molecules (mostly polymers). As mentioned in the previous sections the GNP in this project will be coated with a methoxy-PEG$_n$ thiol layer for optimal targeting to the tumor. This PEG layer immediately aids the stability of the colloid.  Another way to stabilize the colloid is by using electrostatic repulsion. If all particles have an equal and high surface potential the coulomb repulsion will stop them from aggregating.
Measurements of the surface potential is a typical characterization of gold nanoparticles and will be performed in this project.
When no functionalization has happened, the citrate ions (see section \ref{subsubsec:Funct}) surround the particles and the surface potential should be negative. If the particles are coated with an methoxy-PEG$_n$ thiol layer no surface potential should be present.\\
As figure \ref{fig:zeta} shows the naked gold particles in solution have one tightly bound layer of negative ions (citrate ions) and a second less tightly bound layer of positive and negative ions. The surface potential of this second layer will determine the electrostatic properties of the particles, it is called the $\zeta$-potential. In general it is said that if the absolute value of the $\zeta$-potential is larger than $30mV$ the colloid is stable. That is, there is enough electrostatic repulsion to prevent the particles from aggregating. \\
\begin{wrapfigure}{r}{0.45\textwidth}
  \begin{center}
  \includegraphics[width=0.4\textwidth]{gfx/zeta}
  \end{center}
  \caption{Schematic representation of the double layer configuration of particles in a colloid.}
  \label{fig:zeta}
\end{wrapfigure}

To measure the $\zeta$-potential the Laser-Dopler electrophoresis technique is used. When an electric field is applied over a sample the charged particles start to accelerate. Since the particles are in solution they undergo a drag force and will eventually move at a constant velocity $v$.
\begin{equation}
  E\cdot q=\alpha \cdot v \Rightarrow v=\mu_e\cdot E \quad \text{with }\mu_e=\frac{q}{\alpha}
\end{equation}
Here $\mu_e$ is called the electrophoretic mobility and its link with the $\zeta$-potential in the Smoluchowki approximation is given by the following equation.
\begin{equation}
  \zeta=\frac{2 \eta \mu_e}{3\epsilon}=\frac{2 \eta v}{3\epsilon E}
\end{equation}
with $\eta$ the viscosity and $\epsilon$ the dielectric constant of the medium. Measuring the constant velocity of the particles in electric field E is done by irradiating the moving particles with a laser of known wavelength and registrate the doppler shift in the scattered radiation. Hence an experimental value for the $\zeta$-potential is obtained and this is a quantitative indicator for the stability of the colloid.
\subsection{Radiosensitization}
\subsubsection{Sample Preparation}
To verify the effect of Gold nanoparticles on DNA a mixture of DNA and GNP solution (PEGylated) is prepared. These samples will then be irradiated with varying X-Ray dose ($0$Gy-$15$Gy). To have an optimal radiosensitization effect on the DNA irradiation the GNP solutions have to be centrifuged. The concentration of gold nanoparticles in a $250$ ml solution obtained following the protocol described in the appendix is $1.7\cdot10^9$ GNP/$\mu$l, $2.1 \cdot10^8$ GNP/$\mu$l and $6.3 \cdot10^7$ GNP/$\mu$l for the particles of size $15$nm, $30$nm and $45$nm respectively. This has been calculated using the known molar masses of gold and tetrachlorauric acid, the mass density of gold ($\rho_{Au}=19.32$) and under the assumption that all particles have the same size and are spherical.
The total volume is divided in four equal parts and each part is functionalized with a different PEG. After centrifugation each volume is reduced to $0.5$ml achieving a concentration for the $15$nm, $30$nm and $45$nm particles of $2.1\cdot10^{11}$ GNP/$\mu$l, $2.7\cdot10^{10}$ GNP/$\mu$l and $7.9\cdot10^{9}$ GNP/$\mu$l respectively.
The DNA being used is a supercoiled double-stranded plasmid of $5386$ basepair long and a molecular weight of $3.50\cdot10^6$Da \footnote{phiX$174$ from Thermo Scientific}. $2.5\mu$l of $0.5\mu$g/$\mu$l DNA solution is mixed with $30\mu$l of the centrifuged GNP solutions. To prevent the DNA from deterioration $17.5\mu$l of phosphate buffered saline (PBS)\footnote{from Sigma-ALdrich} was added to each sample.  \\
The final samples are a $50\mu$l solution with roughly a $30:1$ GNP:DNA ratio for the $15$nm particles, a $4:1$ ration for the $30$nm particles and a $1:1$ ratio for the $45$nm particles.
\subsubsection{DNA damage analysis}
The irradiation of the samples was done using a Baltograph XSD $225$ X-ray generator from Balteau NTD at 199kVp. The $12$ samples are treated with different doses: $0$Gy, $5$Gy, $10$Gy and $15$Gy. \\
After the irradiation, the damage on the DNA due to the X-ray radiation is analyzed. As discussed in section \ref{subsec:bioeffects} the effect of ionizing radiation on DNA can be a single strand break or a double strand break, but double strand breaks are the most favorable for damaging tumor cells, because their repair mechanisms are the most complex. When a single strand break occurs in supercoiled DNA it becomes circular. A double strand break has linear DNA as a result. \\
The analysis of the DNA damage is done using gel electrophoresis. With this technique it is possible to separate charged macromolecules based on their charge and size. The charged molecules that need to be separated are put into a well at one end of a gel. Then a voltage is applied across the gel and the charged molecules move trough the gel. Smaller molecules and molecules with a larger charge will move faster trough the gel, whereas larger molecules and molecules with a lower charge will move slower. Molecules with the same charge and size will travel the same distance trough the gel and will collect into a band.

\begin{wrapfigure}{l}{0.5\textwidth}
  \begin{center}
    \includegraphics[width=0.48\textwidth]{gfx/afbgel}
  \end{center}
  \caption{An example of a gel electrophoresis result. The DNA bands correspond to circular, linear and supercoiled DNA \cite{afbgel}\cite{supercoiled}.}
	 \vspace{-10pt}
	\label{fig:afbgel}
\end{wrapfigure}

Because DNA is negatively charged, it is suitable for gel electrophoresis. Supercoiled DNA is the most compact and so has the highest mobility, therefore supercoiled DNA travels the largest distance in the gel. Circular DNA, caused by a single strand break, has the lowest mobility and therefore travels the smallest distance trough the gel, due to its large expanse. Linear DNA which is a result of a double strand break travels an intermediate distance. Figure \ref{fig:afbgel} shows an example of a gel electrophoresis image with circular, linear and supercoiled DNA. \\  
Here a voltage of $120$V is applied across an $1\%$ agarose gel during $35$ minutes. Afterward the gel is colored with special dye that binds to the DNA and is active when exposed to UV-light. Then imaging of the gel electrophoresis experiment is done using the ... transilluminator. The amount of DNA damage is analyzed by determining the intensity of the different bands with the use of the program ImageJ. 

%\newpage
\section{Results and Discussion}
\subsection{Core Diameter}
\begin{wraptable}{r}{0.4\textwidth}
  \caption{Core diameter of the particles observed in the TEM images. Exp. Size is the diameter expected from the chemical protocol used}
  \label{tab:TEM}
  \begin{tabular}{ c | c }
    Exp. Size (nm) & Size (nm)\\
    \toprule
    $15$ & $12.98 \pm 0.23$\\
     & $2.99 \pm 0.16$\\
    $30$ & $18.29 \pm 0.23$\\
    $45$ & $46.75 \pm 0.47$
  \end{tabular}
\end{wraptable}
As described in section \ref{subsubsec:CoreDia} TEM images of the particles are created to determine their shape and core diameter. According to the synthesis protocol used (see section \ref{subsubsec:Turk} and appendix) we expect spherical particles.  Several images were created from the three different colloids (expected size $15$nm, $30$nm and $45$nm) in advance of the PEG functionalization, thus naked gold nanoparticles. In figure \ref{fig:TEM} one image of each colloid is shown in combination with a histogram indicating the size distribution. The data in the histogram is a combination of data from several images.\\
The TEM images show approximately spherical nanoparticles. They cluster together due to Van der Waals interactions but do not aggregate due to the electrostatic repulsion of the citrate ions surrounding the naked particles.
The average size calculated from the data is presented in table \ref{tab:TEM}.
For the particles with expected size of $45$nm, the observed average size is slightly bigger and for the smallest particles slightly smaller.%#TODO kleine uitleg waarom de afwijking en waarom we dit wel oke vinden
The particles requiring attention are  the ones with expected size $30$nm. They are more then $10$nm smaller compared to the size the protocol predicts. The reason for this deviation lies most certainly in the experimental handling of the protocol. (Many other experimenters have used the protocol and obtained the expected results) Since the particles are smaller, presumably a higher concentration of citrate is added to the gold solution. This can happen due to wrongful pipetting work or evaporation of the gold solution while bringing it to boiling temperature. The last one seems unlikely since during the heating a condenser was used with a continuous flow of cold water.\\
The size distributions from the two biggest particles, figures \ref{fig:TEM}(b) and (d), indicate a wide spreading around the average.
If more data had been collected a gaussian size distribution would have been expected since the growth of nanoparticles depends various random factors. For the particles of expected size $30$nm the size distribution approximates a normal gaussian distribution whereas the distribution for the biggest particles appear to be negatively skewed. For the smallest particles though, clearly a subpopulation of diameter around $4$nm is present. The reason for this subpopulation is not sure, to little particles were analyzed to draw immediate conclusions. A possibility is that there was an unequal distribution of the citrate in the gold solution due to unsystematic stirring.
\newpage

\begin{figure}[h!]
  \includegraphics[width=\textwidth]{gfx/TEM_res}
  \caption{}
   \label{fig:TEM}
\end{figure}

\newpage

\subsection{UV-Vis}
\begin{figure}
  \includegraphics[width=\textwidth]{gfx/UVVIS_alle}
  \caption{Absorption spectra for naked GNP of expected size $15$nm, $30$nm and $45$nm}
  \label{fig:UV-VIS all naked}
\end{figure}
Figure \ref{fig:UV-VIS all naked} presents the absorption spectra of the naked gold nanoparticles, thus before the functionalization with PEG. The figure clearly indicates the mechanism of the UV-VIS measurements. Bigger particles have a lower surface plasmon frequency hence a bigger wavelength at which maximum absorption occurs.\\
The most important application of the UV-VIS measurements is to determine the necessary proportion of PEG/GNP solution as described in section \ref{subsubsec:RelSize}.
The reasoning to determine the ideal PEG/GNP proportion from the measurements will be explained here for the $20$k PEG. The same reasoning applies for the other PEG's. In figure \ref{fig:UV-VIS 15nm 20k} four absorption spectra (with different PEG/GNP ratio) for the smallest particles ($15$nm) functionalized with $20$k PEG are presented. Both the curves before and after the addition of NaCl (to stimulate aggregation) are shown. The results for the other particles can be found in the appendix. Each time only four (out of ten) absorption spectra are shown for the sake of clarity of the figure. \\
\begin{wraptable}{r}{0.4\textwidth}
  \caption{Necessary PEG/GNP proportions (volume/volume) for each PEG to create a stable colloid. The proportions are valid for a $1$g/l PEG solution and a GNP solution obtained from the protocol described in the appendix. }
  \label{tab:PEG/GNP}
  \begin{tabular}{ c | c }
    PEG & PEG/GNP \\
    \toprule
    $1$k & $4/100$\\
    $5$k& $4/10$\\
    $10$k & $6/10$\\
    $20$k & $8/10$
  \end{tabular}
\end{wraptable}
When no salt is added, the four absorption maxima from the different spectra should coincide and that is also what had been measured, see figure \ref{fig:UV-VIS 15nm 20k}. Even with insufficient PEG-coating the particles will not immediately aggregate together since they are then still slightly coated with negative citrate ions causing additional electrostatic repulsion. When adding NaCl the electrostatic effects will disappear and only the particles coated with enough PEG to provide the necessary steric repulsion will hold their size. Clearly, a $1/10$ ratio is not enough since with NaCl a secondary peak rises at a higher wavelength indicating the presence of bigger particles. The peak of the $4/10$ absorption spectrum broadens upon addition salt indicating still little aggregation. Almost no difference is observed between the $8/10$ and $10/10$ ratio so the ideal proportion is $8/10$ since we look for the minimum necessary PEG/GNP proportion.
\\The determined PEG/GNP proportions for the four different PEG's are listed in table \ref{tab:PEG/GNP}. Remarkable is the fairly low proportion necessary when working with the smallest PEG (length $1$k). When variating the PEG/GNP proportion between $1/10$ and $10/10$ it was observed that the lower the proportion, the better the reaction of the colloid to the salt. Lowering even further this trend remained, see figure \ref{fig:UV-VIS 15nm 1k}. This is strange and contradicts the theory. It is correct that less PEG (in mass) is needed when using the smaller PEG since in $1$g more PEG molecules are present (This is observed for the other PEG's). However, this doesn't explain why when using $1$k PEG there suddenly seems to be a maximum PEG proportion at which the colloid becomes unstable again. Since this is only observed with the smallest PEG it has to have something to do with the length of the PEG chain. In a shorter PEG chain the negative charge of the thiol group becomes more significant. When there's an over-saturation of PEG in the colloid, the negative charges of the thiol groups at the surface of the particles will start to repel each other. This process causes a weaker bound between the PEG and the nanoparticles facilitating aggregation upon NaCl addition. This explanation has not been tested nor experimentally verified due to time limits. It is worth investigating this strange interaction of GNP and $1$k PEG when a stable GNP colloid with smallest particles possible is desired. 
After the necessary PEG/GNP proportion was determined based on the absorption spectra with the smallest particles, control measurements were done with the absorption spectra from the bigger particles. An example with $20$k PEG is shown in figure \ref{fig:UV-VIS all 20k}. This figure indeed indicates that all three colloids don't aggregate after being mixed with NaCl.

\begin{figure}[h!]
	\centering
	\includegraphics[width=\textwidth]{gfx/UVVIS_20k}
	\caption{Absorption spectra for 15nm GNP (expected size) functionalized with 20k PEG for different PEG/GNP proportions, with and without NaCl.}
  \label{fig:UV-VIS 15nm 20k}
\end{figure}

\begin{figure}[h!]
	\centering
	\includegraphics[width=\textwidth]{gfx/UVVIS_alle_20k}
	\caption{Optical density in function of wavelength for 15, 30 and 45nm GNP, with 20k PEG with PEG/GNP=8/10.}
  \label{fig:UV-VIS all 20k}

\end{figure}

\subsection{Zeta potential}
\begin{table}[h!]
  \centering
  \caption{$\zeta$-potential measurement results in mV for three different particles, with and without functionalisation using PEG.}
\begin{tabular}{c||c|c|c|c|c}
  Size (nm) & No PEG  & $1$k PEG & $5$k PEG & $10$k PEG & $20$k PEG\\
    \hline
  \hline
  $15$&$-33.73\pm1.85$&$-13.08\pm1.67$ &$-5.20\pm1.46 $&$-5.87 \pm1.65$&$-4.45\pm1.28$ \\
  $30$&$-20.65 \pm 1.89$ & $4.48 \pm 1.46$ & $0.76 \pm 1.20$ & $1.06 \pm 1.10$ & $-10.62\pm2.09$\\
  $45$&$-24.70 \pm 3.57$ &$-9.05\pm1.32$ & $1.77 \pm 1.60$ & $-3.51 \pm 2.14$ & $-7.90 \pm 1.69$ \\

\end{tabular}
\end{table}

\begin{figure}
  \centering
  \begin{subfigure}[t]{\textwidth}
    \includegraphics[width=\textwidth]{gfx/Zeta_45nm}
    \caption{}
    \label{fig:Zeta_45}
  \end{subfigure}
  \begin{subfigure}[t]{\textwidth}
    \includegraphics[width=\textwidth]{gfx/Zeta_45nm_fit.png}
    \caption{}
    \label{fig:Zeta_45_fit}
  \end{subfigure}
  \caption{$\zeta$-potential measurement results for particles of 45nm without funtionalisation (a) data and (b) Lorentzian fit. Ten measurements were done and the average was calculated (black curve).}
\end{figure}

\subsection{Hydrodynamic Radius (DLS)}
\begin{table}[h!]
  \centering
  \caption{DLS measurement results in nm for three different particles, of expected size $15$nm, $30$nm and $45$nm, with and without functionalisation using PEG. The length of the PEG chain varies from 20k to 1k .}
\begin{tabular}{c || c | c | c | c | c }
  Size (nm) & No PEG  & $1$k PEG & $5$k PEG & $10$k PEG & $20$k PEG\\
  \hline
  \hline
  $15$ &$28.37 \pm 1.53$  & $135.72 \pm 38.01$ & $40.26\pm 1.25$ & $54.03 \pm 0.56$ & $66.00 \pm 4.71$\\
  & $9.70\pm 1.44$& & &$14.05 \pm 4.13$ \\
  $30$ & $30.03 \pm 1.25$ & $99.61 \pm 17.33$ & $45.41 \pm 1.26$ & $53.28 \pm 1.66$ & $61.29 \pm 1.50$\\
  & $4.99 \pm 1.55$ & $25.37 \pm 1.84$ & $7.85 \pm 1.05$ & $5.90 \pm 0.97$ & $7.36 \pm 1.73$\\
  $45$ & $50.75 \pm 2.05$ & $109.54 \pm 0.69$ & $70.04 \pm 3.06$ & $71.56 \pm 1.87$ & $76.14 \pm 0.66$ \\
  & $3.28 \pm 0.79$ & & $10.67 \pm 1.44$ & $10.74 \pm 1.13$ & $5.15 \pm 0.74$ \\

\end{tabular}
\end{table}
\begin{figure}[h!]
  \centering
  \includegraphics[width=0.8\textwidth]{gfx/DlsALl.png}
  \caption{DLS measurement results for three different particles, of expected size $15$nm, $30$nm and $45$nm, before (transparent) and after PEG functionalisation.}
\end{figure}

\newpage
\section{Conclusion}
Synthesizing gold nanoparticles following the Turkevich method is a reliable way to obtain particles of a specific size. Though, one always has to keep in mind that smaller particles are also created and that the system is very sensitive to slight deviations of concentration. To be certain about the size distributions of the particles in a colloid, TEM image analysis is the best and most reliable way. In the $15$nm particle colloid a subpopulation has been found, probably due to unequal distribution of the citrate ions in solution. \\
Throughout all measurements the particles functionalized with $1$k PEG provide strange results. First, there seems to be a maximum PEG/GNP proportion at which point the colloid becomes unstable again and the particles start to aggregate. One would expect that once the whole surface of the particles is coated with a PEG-layer, which provides the necessary steric repulsion to prevent aggregation, adding more PEG does not influence this equilibrium. Secondly, the hydrodynamic radius of the $1$k PEG functionalized particles appears to be larger then the $R_h$ of particles with the same core diameter but functionalized with larger PEG. The $\zeta$-potential results are also questionable and vary from extremely negative to even positive. In the gelelectrophoresis some samples functionalized with $1$k appear to be strongly negatively charged. This could indicate that there is too little neutral PEG and the particles are still functionalized with negative citrate ions.
If the $1$k PEG needs to be used for medical applications further research is needed to account for and eliminate this strange behavior.
\\The characterization results for the other samples fulfill our expectations. In measuring the hydrodynamic radius using the DLS technique though, some technical problems were observed with the laser. This can account for unexpected results and large errors. The measurements have to be interpreted carefully before they are generalized. When ignoring the results for particles functionalized with $1$k PEG, the expected trend that functionalization with larger PEG corresponds to larger hydrodynamic radii is observed. \\
The $\zeta$-potential for the functionalized samples is in general measured to be neutral as it should be due to the methoxy end-group of the PEG. \\
When analyzing the DNA damage after the irradiation it is observed that in general the larger particles have a slightly better effect than smaller particles. This however, as stated before, contradicts the literature \cite{lies}. It is expected that this apparent better effect for larger particles is due to a smaller total gold concentration in the samples with smaller particles resulting from the centrifuging process. Furthermore the particles functionalized with the $5$k PEG (the smallest PEG when ignoring the strange results for $1$k PEG) provide the highest damage. The effect of higher dose should be an increase in damage and this is not consequently observed. Most likely this is due to incorrect storing during the time between the preparation of the samples, irradiation and the gelelectrophoresis. This unfortunate timing of the experiments occurred due to technical problems and should be avoided in further experiments.
%#TODO something about the DNA damage due to the others

\newpage
\section*{Appendix}
\subsection*{Synthesis gold nanoparticles}
To synthesis the gold nanoparticles following protocol is followed.
\begin{enumerate}
  \item Perpare a solution of 100ml $0.01\%$ HAuCl$_4$
  \item Heat the solution till boiling temperature while stirred
  \item Add $2.5$ml, $1.24$ml, $0.8$ml of a $1\%$ Na$_3$C$_6$H$_5$O$_7$ solution for particles of size respectively $15$nm, $30$nm and $45$nm
  \item Let the solution boil for 20 minutes while stirred
  \item Let the solution cool down for at least $60$min while protected from light
  \item Store the solutoin at $4 ^{\circ} C$ and protected from light
\end{enumerate}
\newpage

\subsection*{UV-Vis measurements}
\begin{figure}
  \includegraphics[width=\textwidth]{gfx/UVVIS_1k}
  \caption{Absorption spectra for 15nm GNP (expected size) functionalized with 1k PEG for different PEG/GNP proportions, with and without NaCl.}
  \label{fig:UV-VIS 15nm 1k}
\end{figure}

\begin{figure}
  \includegraphics[width=\textwidth]{gfx/UVVIS_5k}
  \caption{Absorption spectra for 15nm GNP (expected size) functionalized with 5k PEG for different PEG/GNP proportions, with and without NaCl.}
  \label{fig:UV-VIS 15nm 5k}
\end{figure}

\begin{figure}
  \includegraphics[width=\textwidth]{gfx/UVVIS_10k}
  \caption{Absorption spectra for 15nm GNP (expected size) functionalized with 10k PEG for different PEG/GNP proportions, with and without NaCl.}
  \label{fig:UV-VIS 15nm 10k}
\end{figure}


%----------------------------------------------------------------------------------------
%	BIBLIOGRAPHY
%----------------------------------------------------------------------------------------
\newpage
\bibliographystyle{plain}
\bibliography{content/bib}

%----------------------------------------------------------------------------------------


\end{document}
